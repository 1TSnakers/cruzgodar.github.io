\documentclass{article}

\usepackage[dvipsnames]{xcolor}

\usepackage{tikz-cd}

\usepackage[utf8]{inputenc}
\usepackage{amsmath}
\usepackage{amsfonts}
\usepackage{amssymb}
\usepackage[margin=1in]{geometry}

\usepackage{graphicx}
\graphicspath{{graphics/}}

\usepackage{scalerel}

\usepackage{stmaryrd}

\usepackage{MnSymbol}

\usepackage{mdframed}

\usepackage{titlesec}

\usepackage{hyperref}

\hypersetup
{
	colorlinks = true,
	urlcolor = OliveGreen
}

\titleformat{\section}
{\normalfont \Large \bfseries \centering}{\Roman{section} --- }{0pt}{}




\definecolor{DefGreen}{rgb}{0,0.5,0}
\definecolor{TheoremOrange}{rgb}{0.88,0.6,0.08}
\definecolor{LemmaYellow}{rgb}{1,1,0}
\definecolor{CorollaryBlue}{rgb}{0,0.29,0.77}
\definecolor{PropPink}{rgb}{1,0,.75}
\definecolor{ProofPurple}{rgb}{0.58,0,1}
\definecolor{AxiomRed}{rgb}{1,0,0}



\usepackage{amsthm}

\newtheoremstyle{colontheorem}
	{0in}                    	% Space above
	{.15in}                   	% Space below
	{\normalfont}      		    % Body font
	{}                          % Indent amount
	{\bfseries}                 % Theorem head font
	{:}                         % Punctuation after theorem head
	{.5em}                      % Space after theorem head
	{}							% Theorem head spec (can be left empty, meaning ‘normal’)
	
\theoremstyle{colontheorem}

\newtheorem{theorem}{Theorem}[section]
\newtheorem{proposition}[theorem]{Proposition}
\newtheorem{definition}[theorem]{Definition}
\newtheorem{axiom}[theorem]{Axiom}

\newtheorem{lemma}{Lemma}[theorem]
\newtheorem{corollary}{Corollary}[theorem]

\newtheorem{exercise}{Exercise}[section]



\newcommand{\fadeline}
{
	\noindent\begin{tikzpicture}[baseline]
		\path[left color=white,right color=white,middle color=black]
		(0,0) rectangle (\textwidth,.5pt);%
	\end{tikzpicture}
}


\newcommand{\Span}{\textnormal{span}}
\newcommand{\Null}{\textnormal{null }}
\newcommand{\Range}{\textnormal{range }}



\newenvironment{Theorem}
{
	\begin{mdframed}[backgroundcolor=TheoremOrange!10]
	\begin{theorem}
}
{
	\end{theorem}
	\end{mdframed}
	
	\vspace{.15in}
}

\newenvironment{Proposition}
{
	\begin{mdframed}[backgroundcolor=PropPink!10]
	\begin{proposition}
}
{
	\end{proposition}
	\end{mdframed}
	
	\vspace{.15in}
}

\newenvironment{Def}
{
	\begin{mdframed}[backgroundcolor=DefGreen!10]
	\begin{definition}
}
{
	\end{definition}
	\end{mdframed}
	
	\vspace{.15in}
}

\newenvironment{Axiom}
{
	\begin{mdframed}[backgroundcolor=AxiomRed!10]
	\begin{axiom}
}
{
	\end{axiom}
	\end{mdframed}
	
	\vspace{.15in}
}

\newenvironment{Lemma}
{
	\begin{mdframed}[backgroundcolor=LemmaYellow!10]
	\begin{lemma}
}
{
	\end{lemma}
	\end{mdframed}
	
	\vspace{.03in}
}

\newenvironment{Corollary}
{
	\begin{mdframed}[backgroundcolor=CorollaryBlue!10]
	\begin{corollary}
}
{
	\end{corollary}
	\end{mdframed}
	
	\vspace{.09in}
}

\newenvironment{Proof}
{
	\begin{mdframed}[backgroundcolor=ProofPurple!10]
	\textbf{Proof:}%
}
{
	\end{mdframed}
	
	\vspace{.085in}
}

\newenvironment{Example}
{
	\begin{mdframed}
	\textbf{Example:}%
}
{
	\end{mdframed}
	
	\vspace{.15in}
}



\setlength{\parindent}{0pt}




\begin{document}

\vspace*{.5in}

\begin{center}
	\Huge Linear Algebra Notes\\
	
	\vspace{.25in}
	
	\Large Cruz Godar\\
	
	\vspace{.25in}
	
	\normalsize Math 306 and Math 406 --- Professor Mendes
\end{center}

\vspace{.5in}





\begin{center}
	\section{Vector Spaces}
	\vspace{.1in}
\end{center}



\begin{Def}
	
	Let $k$ be a field. A \textbf{vector space} over $k$ is a set $V$ equipped with two binary operations $+$ and $\cdot$ such that for all $u,v,w \in V$ and $c,d \in k$,
	
	\begin{enumerate}
		
		\item $u+v \in V$ and $cv \in V$.
		\item $u+v = v+u$.
		\item $u+(v+w) = (u+v)+w$ and $c(dv) = (cd)v$.
		\item $c(u+v) = cu+cv$ and $(c+d)v = cv+dv$.
		\item There is a vector $0 \in V$ that satisfies $v+0 = v$ for all $v \in V$.
		\item There is a vector $1 \in V$ that satisfies $1v = v$ for all $v \in V$.
		\item For all $v \in V$, there is a vector $-v \in V$ such that $v + (-v) = 0$.
		
	\end{enumerate}
	
\end{Def}



\begin{Proposition}
	
	The element $0$ is unique.
	
	\begin{Proof}
		%
		Suppose there were two elements $0, 0'$ satisfying $v + 0 = v + 0' = v$ for all $v \in V$. Then $0 + 0' = 0$, but $0 + 0' = 0' + 0 = 0'$, so $0 = 0'$. 
		
	\end{Proof}
	
\end{Proposition}



\begin{Proposition}
	
	For each $v \in V$, $-v$ is unique.
	
	\begin{Proof}
		%
		Suppose there were two elements $-v, (-v)'$ satisfying $v + (-v) = v + (-v)' = 0$. Then $-v = -v + (v+(-v)')$, so $-v = (-v)'$.
		
	\end{Proof}
	
\end{Proposition}



\begin{Proposition}
	
	For all $v \in V$, $0v = 0$ and $(-1)v = -v$.
	
	\begin{Proof}
		%
		We have 
		\begin{align*}
		0 &= 0v + (-0v)\\
		&= (0 + 0)v + (-0v)\\
		&= 0v + (0 + (-0))v\\
		&= 0v
		\end{align*} and \begin{align*}
		(-1)v &= (-1)v + 0\\
		&= (-1)v + v + (-v)\\
		&= (-1 + 1)v + (-v)\\
		&= -v.
		\end{align*}
		
	\end{Proof}
	
\end{Proposition}



\begin{Def}
	
	Let $V$ be a vector space. A \textbf{subspace} of $V$ is a nonempty set $U \subseteq V$ such that
	
	\begin{enumerate}
		
		\item $u+v \in U$ for all $u,v \in U$.
		\item $cu \in U$ for all $u \in U$ and $c \in k$.
		
	\end{enumerate}
	
\end{Def}



\begin{Def}
	
	Let $V$ be a vector space and $U$ and $V$ subspaces of $V$. The sum of $U$ and $V$ is $U+W = \{u+w\ |\ u \in U, w \in W\}$. If each element of $V$ can be expressed uniquely as an element of $U+W$, we say that $U+W$ is a \textbf{direct sum}, and we write $U \oplus W$.
	
\end{Def}



\begin{Proposition}
	
	$U+W$ is a direct sum if and only if the only expression of $0$ in $U+W$ is $0+0$.
	
	\begin{Proof}
		%
		($\Rightarrow$) If $U+W$ is direct, then since $0 = 0+0$ is one expression of $0$, it must be the only one.\\
		
		($\Leftarrow$) Let $v \in U+W$ and suppose $v = u + w = u' + w'$ for some $u, u' \in U$ and $w, w' \in W$. Then $0 = v - v = (u+w) + (u'+w') = (u-u') + (w-w')$. Thus $u-u' = w-w' = 0$, so $u = u'$ and $w = w'$. 
		
	\end{Proof}
	
\end{Proposition}



\begin{Proposition}
	
	$U+W$ is a direct sum if and only if $U \cap W = \{0\}$.
	
	\begin{Proof}
		%
		($\Rightarrow$) Suppose $U+W$ is direct and let $v \in U \cap W$. Then $0 = v + (-v)$, so by the previous result, $v = -v = 0$.\\
		
		($\Leftarrow$) Assume $U \cap W = \{0\}$ and suppose $u + w = 0$. Then $u = -w \in W$, so $u \in U \cap W$ and is therefore $0$. Thus $u = w = 0$, so the previous proposition gives that $U + W$ is direct.
		
	\end{Proof}
	
\end{Proposition}





\begin{center}
	\vspace{.25in}
	\fadeline
	\vspace{.25in}
	
	\section{Bases and Dimension}
	
	\vspace{.1in}
\end{center}



\begin{Def}
	
	Vectors $v_1, ..., v_n \in V$ are \textbf{linearly independent} if $c_1 v_1 + \cdots + c_n v_n = 0$ for $c_i \in k$ implies $c_1 = \cdots = c_n = 0$, and \textbf{linearly dependent} if not (that is, $c_1 v_1 + \cdots + c_n v_n = 0$ for some $c_i \in k$ not all zero).
	
\end{Def}



\begin{Def}
	
	The \textbf{span} of $v_1, ..., v_n \in V$ is the set $\Span \{v_1, ..., v_n\} = \{c_1 v_1 + \cdots + c_n v_n\ |\ c_i \in k\}$. A set of vectors $\{v_1, ..., v_n\} \subseteq V$ \textbf{spans} $V$ if $\Span \{v_1, ..., v_n\} = V$.
	
\end{Def}



\begin{Proposition}
	
	Let $v_1, ..., v_n \in V$. Then $\Span \{v_1, ..., v_n\}$ is a subspace of $V$.
	
	\begin{Proof}
		%
		First, $0 \in \Span \{v_1, ..., v_n\}$, so the set is nonempty. Next, $(c_1 v_1 + \cdots + c_n v_n) + (d_1 v_1 + \cdots + d_n v_n) = (c_1 + d_1) v_1 + \cdots + (c_n + d_n) v_n \in \Span \{v_1, ..., v_n\}$, so the set is closed under addition, and finally, $c(c_1 v_1 + \cdots + c_n v_n) = (c c_1) v_1 + \cdots + (c c_n) v_n \in \Span \{v_1, ..., v_n\}$, so it is closed under scalar multiplication. Thus $\Span \{v_1, ..., v_n\}$ is a subspace of $V$.
		
	\end{Proof}
	
\end{Proposition}



\begin{Def}
	
	A \textbf{basis} for a vector space $V$ is a set of vectors $\{v_1, ..., v_n\} \subseteq V$ that are linearly independent and span $V$.
	
\end{Def}



\begin{Def}
	
	A vector space is \textbf{finite-dimensional} if it has a finite basis.
	
\end{Def}



\begin{Theorem}
	
	Let $V$ be a finite-dimensional vector space. If $v_1, ..., v_k \in V$ are linearly independent, then they can be extended to form a basis for $V$.
	
	\begin{Proof}
		%
		Suppose $\Span \{w_1, ..., w_n\} = V$. Then $\Span \{v_1, ..., v_k, w_1, ..., w_n\} = V$, so if $v_1, ..., v_k, w_1, ..., w_n$ are linearly independent, $\{v_1, ..., v_k, w_1, ..., w_n\}$ is a basis. Otherwise, $c_1 v_1 + \cdots + c_k v_k + d_1 w_1 + \cdots + d_n w_n = 0$ for some $c_i , d_i \in k$. Not all the $d_i$ can be zero, since then $v_1, ..., v_k$ would be linearly dependent, so there is a $d_j \neq 0$. Then $\Span \{v_1, ..., v_k, w_1, ..., w_{j-1}, w_{j+1}, ..., w_n\} = \Span \{v_1, ..., v_k, w_1, ..., w_n\}$, since we can create $w_j$ from the other vectors. Continue removing $w_i$ until the set is linearly independent (this will terminate, since at most we will have $v_1, ..., v_k$ once again).
		
	\end{Proof}
	
\end{Theorem}



\begin{Theorem}
	
	Every basis for a finite-dimensional vector space has the same number of elements.
	
	\begin{Proof}
		%
		Let $\{v_1, ..., v_n\}$ and $\{w_1, ..., w_m\}$ be bases for $V$. Then by definition, $v_1 = c_1 w_1 + \cdots + c_m w_m$ for some $c_i \in k$ not all zero. Without loss of generality, assume $c_1 \neq 0$. Then $w_1 = (-c_1)^{-1}(-v_1 + c_2 w_2 + \cdots + c_m w_m) \in \Span \{v_1, w_2, ..., w_m\}$, so $\Span \{v_1, w_2, ..., w_m\} = \Span \{w_1, ..., w_m\} = V$. Repeat this process until we have $V = \Span \{v_1, ..., v_n, w_{n+1}, ..., w_m\}.$ If $n > m$, then $V = \Span \{v_1, ..., v_m\}$, but then $v_1, ..., v_n$ are not linearly independent. Thus $n \leq m$, and repeating the proof by eliminating the $v_i$ gives that $m \leq n$, so $n = m$.
		
	\end{Proof}
	\vspace{-.02in}
\end{Theorem}



\begin{Def}
	
	Let $V$ be a vector space. The \textbf{dimension} of $V$, denoted $\dim V$, is the number of elements in a basis for it.
	
\end{Def}



\begin{Proposition}
	
	If $v_1, ..., v_n \in V$ are linearly independent and $\dim V = n$, then $\{v_1, ..., v_n\}$ is a basis for $V$.
	
	\begin{Proof}
		%
		Suppose not. Extend $\{v_1, ..., v_n\}$ to form a basis for $V$. But then that basis would have more than $n$ elements. $\lightning$
	\end{Proof}
	
\end{Proposition}



\begin{Proposition}
	
	If $v_1, ..., v_n \in V$ span $V$ and $\dim V = n$, then $\{v_1, ..., v_n\}$ is a basis for $V$.
	
\end{Proposition}





\begin{center}
	\vspace{.25in}
	\fadeline
	\vspace{.25in}
	
	\section{Linear Maps}
	
	\vspace{.1in}
\end{center}



\begin{Def}
	
	Let $V$ and $W$ be vector spaces. A \textbf{linear map} from $V$ to $W$ is a function $T: V \longrightarrow W$ such that
	
	\begin{enumerate}
		
		\item For all $u, v \in V$, $T(u+v) = Tu + Tv$.
		\item For all $u \in V$ and $c \in k$, $T(cu) = cTu$.
		
	\end{enumerate}

	We write $Tu$ to mean $T(u)$. The set of all linear maps from $V$ to $W$ is denoted $\mathcal{L}(V, W)$.
	
\end{Def}



\begin{Proposition}
	
	$\mathcal{L}(V, W)$ is a vector space under function addition and composition.
	
\end{Proposition}



\begin{Proposition}
	
	Let $\{v_1, ..., v_n\}$ be a basis for $V$ and let $w_1, ..., w_n \in W$. Then there is a unique linear map $T \in \mathcal{L}(V,W)$ such that $Tv_i = w_i$ for each $i$.
	
	\begin{Proof}
		%
		Such a $T$ exists, since we can define it by $T(c_1 v_1 + \cdots + c_n v_n) = c_1 w_1 + \cdots + c_n w_n$, and it follows that every linear map $S$ with $Sv_i = w_i$ is equal to $T$ by the properties of linear maps.
		
	\end{Proof}
	
\end{Proposition}



\begin{Def}
	
	Let $T \in \mathcal{L}(V,W)$. The \textbf{null space} of $T$ is the set $\Null T = \{x \in V\ |\ Tx = 0\}$, and the \textbf{range} of $T$ is the set $\Range T = \{Tv\ |\ v \in V\}$.
	
\end{Def}



\begin{Proposition}
	
	Let $T \in \mathcal{L}(V,W)$. Then $\Null T$ is a subspace of $V$ and $\Range T$ is a subspace of $W$.
	
\end{Proposition}



\begin{Theorem}
	
	\textbf{(The Fundamental Theorem of Linear Maps)} Let $T \in \mathcal{L}(V,W)$. Then $\dim V = \dim \Null T + \dim \Range T$.
	
	\begin{Proof}
		%
		Let $\{v_1, ..., v_k\}$ be a basis for $\Null T$ and extend it to $\{v_1, ..., v_n\}$ to form a basis for $V$. We claim that $\{Tv_{k+1}, ..., Tv_n\}$ is a basis for $\Range T$.\\
		
		Suppose $c_{k+1} Tv_{k+1} + \cdots + c_n Tv_n = 0$. Then $T(c_{k+1} v_{k+1} + \cdots + c_n v_n) = 0$, so $c_{k+1} v_{k+1} + \cdots + c_n v_n \in \Null T$. Since $\{v_1, ..., v_k\}$ is a basis for $\Null T$, $c_{k+1} v_{k+1} + \cdots + c_n v_n = c_1 v_1 + \cdots + c_k v_k$ for some $c_1, ..., c_k \in k$. Since $v_1, ..., v_n$ are linearly independent, $c_1 = \cdots = c_n = 0$, so in particular, $c_{k+1} = \cdots + c_n = 0$. Thus $Tv_{k+1}, ..., Tv_n$ are linearly independent.\\
		
		Let $w \in \Range T$. Then $T(c_1 v_1 + \cdots + c_n v_n) = w$ for some $c_1, ..., c_n \in k$, and since $c_1 v_1 + \cdots + c_k v_k \in \Null T$, $T(c_1 v_1 + \cdots + c_k v_k) = 0$, so $T(c_{k+1} v_{k+1} + \cdots + c_n v_n) = w$. Then $w = c_{k+1} Tv_{k+1} + \cdots + c_n Tv_n$, so $w \in \Span \{Tv_{k+1}, ..., Tv_n\}$. Thus $Tv_{k+1}, ..., Tv_n$ span $\Range T$.\\
		
		Thus $\{Tv_{k+1}, ..., Tv_n\}$ is a basis for $\Range T$, so in particular, $\dim \Range T = n - k$ and $\dim V = n = \dim \Null T + \dim \Range T = k + (n-k)$.
		
	\end{Proof}
	
\end{Theorem}



\begin{Proposition}
	
	A linear map $T \in \mathcal{L}(V,W)$ is injective if and only if $\Null T = \{0\}$.
	
	\begin{Proof}
		%
		$(\Rightarrow)$ Let $x \in \Null T$. Then $Tx = T0 = 0$, so $x = 0$, since $T$ is injective.\\
		
		$(\Leftarrow)$ Suppose $Tu = Tv$ for $u, v \in V$. Then $T(u-v) = 0$, so $u - v = 0$, since $\Null T = \{0\}$. Thus $T$ is injective.
		
	\end{Proof}
	
\end{Proposition}



\begin{Proposition}
	
	Let $T \in \mathcal{L}(V,W)$. If $\dim V > \dim W$, then $\Null T \neq 0$.
	
	\begin{Proof}
		%
		$\dim \Null T = \dim V - \dim \Range T \geq \dim V - \dim W > 0$.
		
	\end{Proof}
	
\end{Proposition}



\begin{Def}
	
	Let $T \in \mathcal{L}(V,W)$. An \textbf{inverse linear map} to T is a $T^{-1} \in \mathcal{L}(V,W)$ such that $T^{-1}T = I_V$ and $TT^{-1} = I_W$. If such a $T^{-1}$ exists, we call $T$ \textbf{invertible}.
	
\end{Def}



\begin{Proposition}
	
	Let $T \in \mathcal{L}(V,W)$. Then $T^{-1}$ is unique.
	
	\begin{Proof}
		%
		Suppose $T^{-1}_1$ and $T^{-1}_2$ are both inverses to $T$. Then $T^{-1}_1 = T^{-1}_1 T T^{-1}_2 = T^{-1}_2$.
		
	\end{Proof}
	
\end{Proposition}



\begin{Def}
	
	An \textbf{isomorphism} from $V$ to $W$ is an invertible linear map $T \in \mathcal{L}(V,W)$. $V$ and $W$ are \textbf{isomorphic}, denoted $V \simeq W$, if there is an isomorphism from $V$ to $W$.
	
\end{Def}



\begin{Proposition}
	
	If $\dim V = \dim W$, then $V \simeq W$.
	
	\begin{Proof}
		%
		Let $\{v_1, ..., v_n\}$ and $\{w_1, ..., w_n\}$ be bases for $V$ and $W$ and define $T \in \mathcal{L}(V,W)$ by $Tv_i = w_i$. Then $T$ is injective and surjective, so it is an isomorphism.
		
	\end{Proof}
	
\end{Proposition}



\begin{Def}
	
	Let $\{v_1, ..., v_n\}$ and $\{w_1, ..., w_m\}$ be bases for $V$ and $W$ and let $T \in \mathcal{L}(V,W)$. The \textbf{matrix} of $T$ with respect to the chosen bases is the $m \times n$ rectangle of numbers
	
	$$
	M(T) = \begin{bmatrix}
	
	| &   & |\\
	Tv_1 & \cdots & Tv_n\\
	| &   & |
	
	\end{bmatrix} = \begin{bmatrix}
		
		c_{11} & \cdots & c_{1n}\\
		\vdots & \ddots & \vdots\\
		c_{m1} & \cdots & c_{mn}
		
	\end{bmatrix}
	$$
	
	where $Tv_i$ = $c_{1i} w_1 + \cdots + c_{mi} w_m$. Notice that if $v = c_1 v_1 + \cdots + c_n v_n$, then
	
	$$
	M(T) \begin{bmatrix}
	
		c_1\\
		\vdots\\
		c_n
	
	\end{bmatrix} = \begin{bmatrix}
	
	| &   & |\\
	Tv_1 & \cdots & Tv_n\\
	| &   & |
	
	\end{bmatrix} \begin{bmatrix}
	
	c_1\\
	\vdots\\
	c_n
	
	\end{bmatrix} = c_1 Tv_1 + \cdots + c_n Tv_n = Tv.
	$$
	
\end{Def}



\begin{Def}
	
	Let $A \in M_{m, n}(k)$ and $v_1, ..., v_k \in k^n$. We define \textbf{matrix multiplication} by
	
	$$
	A \begin{bmatrix}
	
	| &   & |\\
	v_1 & \cdots & v_k\\
	| &   & |
	
	\end{bmatrix} = \begin{bmatrix}
	
	| &   & |\\
	Av_1 & \cdots & Av_n\\
	| &   & |
	
	\end{bmatrix}.
	$$
	
	Notice that if $T \in \mathcal{L}(V,W)$ and $S \in \mathcal{L}(W,U)$, then $M(S) M(T) = M(ST)$.
	
\end{Def}




\begin{Theorem}
	
	Let $V$ and $W$ be vector spaces. Then $\dim \mathcal{L}(V,W) = (\dim V)(\dim W)$.
	
	\begin{Proof}
		%
		Let $\{v_1, ..., v_n\}$ and $\{w_1, ..., w_m\}$ be bases for $V$ and $W$. We claim that $\mathcal{L}(V,W) \simeq M_{n, m}(k)$. Define $M : \mathcal{L}(V,W) \longrightarrow M_{n, m}(k)$ by sending a linear map to its matrix.\\
		
		($\hookrightarrow$) Suppose $M(T) = M(S)$. Then the columns of each are equal, so $Tv_i = Sv_i$ for all $i$. Thus $T = S$, so $M$ is injective.\\
		
		($\twoheadrightarrow$) Let $A \in M_{n, m}(k)$ and define $T \in \mathcal{L}(V,W)$ by $Tv_i = c_1 w_1 + \cdots + c_m w_m$, where the $c_j$ form the $i$th column of $A$. Then $M(T) = A$, so $M$ is surjective.
		
	\end{Proof}
	
\end{Theorem}



\begin{Def}
	
	Let $V$ be a vector space. The \textbf{dual space} to $V$ is the vector space $V' = \mathcal{L}(V, k)$. By the previous theorem, $\dim V' = \dim V$.
	
\end{Def}



\begin{Def}
	
	Let $\{v_1, ..., v_n\}$ be a basis for $V$. The dual basis to $\{v_1, ..., v_n\}$ is $\{\varphi_{v_1}, ..., \varphi_{v_n}\}$, where
	
	$$
	\varphi_{v_i}(v_j) = \begin{cases} 
	1, & j = i \\
	0, & j \neq i
	\end{cases}.
	$$
	
\end{Def}



\begin{Proposition}
	
	Let $\{v_1, ..., v_n\}$ be a basis for $V$. Then $\{\varphi_{v_1}, ..., \varphi_{v_n}\}$ is a basis for $V'$.
	
\end{Proposition}



\begin{Def}
	
	Let $T \in \mathcal{L}(V,W)$. The dual map $T' \in \mathcal{L}(W',V')$ is defined by $T'\varphi = \varphi T$.
	
\end{Def}



\begin{Theorem}
	
	Let $T \in \mathcal{L}(V,W)$. Then $M(T') = M(T)^\textnormal{T}$.
	
	\begin{Proof}
		%
		$M(T')_{ij}$ is the coefficient of $\varphi_{v_i}$ in $T'\varphi_{w_j} = \varphi_{w_j} T$. If $\varphi_{w_j} T = c_1 \varphi_{v_1} + \cdots + c_n \varphi_{v_n}$, then $M(T')_{ij} = c_i$. But by the definition of $\varphi_{w_j}$, $\varphi_{w_j} Tv_i$ is the coefficient of $w_j$ in the expression of $Tv_i$, which is the definition of $M(T)_{ji}$. Thus $M(T)_{ji} = M(T')_{ij}$, so $M(T') = M(T)^\textnormal{T}$.
		
	\end{Proof}
	
\end{Theorem}



\begin{Def}
	
	Let $U \subseteq V$ (not necessarily a subspace). The \textbf{annihilator} of $U$ is the set $U^0 = \{\varphi \in V'\ |\ \varphi u = 0\ \textnormal{for all}\ u \in U\}$.
	
\end{Def}



\begin{Proposition}
	
	Let $U$ be a subspace of $V$. Then $\dim V = \dim U + \dim U^0$.
	
	\begin{Proof}
		%
		Let $\{v_1, ..., v_k\}$ be a basis for $U$ and extend it to $\{v_1, ..., v_n\}$ to form a basis for $V$. Then $\{\varphi_{v_1}, ..., \varphi_{v_n}\}$ is a basis for $V'$, so $\varphi_{v_{k+1}}, ..., \varphi_{v_n}$ are linearly independent. Since $\Span \{\varphi_{v_{k+1}}, ..., \varphi_{v_n}\} = U^0$, $\dim U^0 = n-k$, so $\dim V = n = \dim U + \dim U^0 = k + (n - k)$.
		
	\end{Proof}
	
\end{Proposition}



\begin{Proposition}
	
	Let $T \in \mathcal{L}(V,W)$. Then $\Null T' = (\Range T)^0$.
	
	\begin{Proof}
		%
		We have $\varphi \in \Null T'$ if and only if $T'\varphi = 0$, if and only if $\varphi T = 0$, if and only if $\varphi Tv = 0$ for all $v \in V$, if and only if $\varphi w = 0$ for all $w \in \Range T$, if and only if $\varphi \in (\Range T)^0$.
		
	\end{Proof}
	
\end{Proposition}



\begin{Proposition}
	
	Let $T \in \mathcal{L}(V,W)$. Then $\Range T' = (\Null T)^0$.
	
\end{Proposition}



\begin{Proposition}
	
	Let $T \in \mathcal{L}(V,W)$. Then $T'$ is injective if and only if $T$ is surjective.
	
	\begin{Proof}
		%
		$T'$ is injective if and only if $\Null T' = \{0\}$, if and only if $(\Range T)^0 = \{0\}$, if and only if $\dim\  (\Range T)^0 = 0$, if and only if $\dim \Range T = \dim W$, if and only if $\Range T = W$.
		
	\end{Proof}
	\vspace{.1in}
	\begin{Corollary}
		
		Let $T \in \mathcal{L}(V,W)$. Then $\dim \Range T' = \dim \Range T$.
		
	\end{Corollary}
	
\end{Proposition}





\begin{center}
	\vspace{.25in}
	\fadeline
	\vspace{.25in}
	
	\section{Eigenvalues and Eigenvectors}
	
	\vspace{.1in}
\end{center}



\begin{Def}
	
	An \textbf{eigenvalue} of a linear map $T \in \mathcal{L}(V) = \mathcal{L}(V,V)$ is an element $\lambda \in k$ such that $Tv = \lambda v$ for some nonzero $v \in V$. This $v$ is called the \textbf{eigenvector} corresponding to $\lambda$.
	
\end{Def}



\begin{Proposition}
	
	Let $T \in \mathcal{L}(V)$ and $\lambda \in k$. Then $\lambda$ is an eigenvalue of $T$ if and only if $T - \lambda I$ is not invertible.
	
	\begin{Proof}
		%
		We have that $\lambda$ is an eigenvalue of $T$ if and only if $Tv = \lambda v$ for some $v \neq 0$, if and only if $(T - \lambda I)v = 0$ for some $v \neq 0$, if and only if $\Null (T - \lambda I) \neq \{0\}$, if and only if $T - \lambda I$ is not invertible.
		
	\end{Proof}
	
\end{Proposition}



\begin{Theorem}
	
	If $\lambda_1, ..., \lambda_k$ are distinct eigenvalues of $T$, then the corresponding eigenvectors $v_1, ..., v_k$ are linearly independent.
	
	\begin{Proof}
		%
		Suppose not. Then there is a minimum $j$ for which $v_1, ..., v_j$ are linearly dependent, so $v_j = c_1 v_1 + \cdots + c_{j-1} v_{j-1}$ for some $c_1, ..., c_{j-1} \in k$. Then
		
		$$
		\lambda_j v_j = \lambda_j (c_1 v_1 + \cdots + c_{j-1} v_{j-1}) = c_1 \lambda_j v_1 + \cdots + c_{j-1} \lambda_j v_{j-1}.
		$$
		
		But we also have
		
		$$
		\lambda_j v_j = Tv_j = T(c_1 v_1 + \cdots + c_{j-1} v_{j-1}) = c_1 \lambda_1 v_1 + \cdots + c_{j-1} \lambda_{j-1} v_{j-1},
		$$
		
		so
		
		$$
		c_1(\lambda_1 - \lambda_j)v_1 + \cdots + c_1(\lambda_{j-1} - \lambda_j)v_{j-1} = 0.
		$$
		
		Since $j$ was minimal, $v_1, ..., v_{j_1}$ are linearly independent, so $c_i(\lambda_i - \lambda_j) = 0$ for all $i \in \{1, ..., j-1\}$. Not every $c_i = 0$, since then $v_j = 0$, so some $c_i \neq 0$, and therefore $\lambda_i = \lambda_j$. But then the eigenvalues are not distinct. $\lightning$
		
	\end{Proof}
	
\end{Theorem}



\begin{Def}
	
	A linear map $T \in \mathcal{L}(V)$ is \textbf{diagonalizable} if there is a basis of eigenvectors of $T$ for $V$ --- that is, a basis such that
	
	$$
	M(T) = \begin{bmatrix}
		\lambda_1 & \cdots & 0\\
		\vdots & \ddots & \vdots\\
		0 & \cdots & \lambda_n
	\end{bmatrix}.
	$$
	
\end{Def}



\begin{Proposition}
	
	If $\dim V = n$ and $T \in \mathcal{L}(V)$ has $n$ distinct eigenvalues, then $T$ is diagonalizable.
	
\end{Proposition}



\begin{Def}
	
	A matrix $A \in M_n(k)$ is \textbf{upper triangular} if it has the form
	
	$$
	A = \begin{bmatrix}
		* & * & \cdots & * & *\\
		0 & * & \cdots & * & *\\
		\vdots & \vdots & \ddots & \vdots & \vdots\\
		0 & 0 & \cdots & * & *\\
		0 & 0 & \cdots & 0 & *
	\end{bmatrix},
	$$
	
	where the $*$ are elements of $k$.
	
\end{Def}



\begin{Def}
	
	Let $T \in \mathcal{L}(V)$. A subspace $U$ of $V$ is \textbf{$\mathbf{T}$-invariant} if $Tu \in U$ for all $u \in U$.
	
\end{Def}



\begin{Theorem}
	
	Let $V$ be a vector space over an algebraically closed field $k$ with $\dim V = n$ and let $T \in \mathcal{L}(V)$. Then there is a basis for $V$ such that $M(T)$ is upper triangular.
	
	\begin{Proof}
		%
		We will proceed by induction. The base case is trivial, since every $1 \times 1$ matrix is upper triangular.\\
		
		Assume that every linear map in $\mathcal{L}(V)$ has such a basis if $\dim V < n$. Let $T \in \mathcal{L}(V)$ and let $\lambda$ be an eigenvalue of $T$ (This exists, since we can choose any basis for $V$ and perform elementary row operations on $M(T)$ to eliminate every element of a non-leading-zero column below the top one). Let $U = \Range (T - \lambda I)$. Then $U$ is $T$-invariant, since $T(Tv - \lambda v) = T(Tv) - \lambda (Tv) \in U$, so $T|_U \in \mathcal{L}(U)$. Since the eigenvector corresponding to $\lambda$ is an element of $\Null (T - \lambda I)$, $U \neq V$. Thus $\dim U < \dim V$, so by assumption, there is a basis $\{u_1, ..., u_k\}$ for $U$ such that $M(T|_U)$ is upper triangular. Extend this to $\{u_1, ..., u_k, v_1, ..., v_j\}$ to form a basis for $V$. Then $Tv_i = Tv_i - \lambda v_i + \lambda v_i = c_1 u_1 + \cdots + c_k u_k + \lambda v_i$ for some $c_1, ..., c_k \in k$, and so
		
		$$
		M(T) = \begin{bmatrix}
			T & *\\
			0 & \lambda I\\
		\end{bmatrix},
		$$
		
		where $T$ is a $k \times k$ upper triangular matrix, $*$ is unspecified, $0$ is the zero matrix, and $\lambda I$ is a $j \times j$ diagonal matrix. Thus $M(T)$ is upper triangular.
		
	\end{Proof}
	
\end{Theorem}



\begin{Theorem}
	
	If $M(T)$ is upper triangular with respect to the basis $v_1, ..., v_n$ and has diagonal entries $\lambda_1, ..., \lambda_n$, then $T$ is invertible if and only if no $\lambda_i = 0$.
	
	\begin{Proof}
		%
		$(\Rightarrow)$ Assume $T^{-1}$ exists and suppose some $\lambda_i = 0$. Let $U = \Span \{v_1, ..., v_i\}$. Then $U$ is $T$-invariant, but $T|_U$ is not surjective, so it is not invertible, and therefore neither is $T$. $\lightning$\\
		
		$(\Leftarrow)$ It is enough to show that $\Null T = \{0\}$, so suppose $T(c_1 v_1 + \cdots + c_n v_n) = 0$. Then $c_1 Tv_1 + \cdots + c_n Tv_n = 0$. Since $Tv_i \in \Span \{v_1, ..., v_i\}$, $c_n = 0$, since $v_n appears only in Tv_n$ and $\lambda_n \neq 0$. Similarly, $c_1 = \cdots = c_{n-1} = 0$. Thus $\Null T = \{0\}$.
		
	\end{Proof}
	
\end{Theorem}



\begin{Theorem}
	
		If $M(T)$ is upper triangular with diagonal entries $\lambda_1, ..., \lambda_n$, then $T$ has eigenvalues $\lambda_1, ..., \lambda_n$.
		
		\begin{Proof}
			%
			If $\lambda$ is an eigenvalue of $T$, then $T - \lambda I$ is not invertible. Then $\lambda_i - \lambda = 0$ for some $i$, since
			
			$$
			M(T - \lambda I) = \begin{bmatrix}
				\lambda_1 - \lambda & * & \cdots & *\\
				0 & \lambda_2 - \lambda & \cdots & *\\
				\vdots & \vdots & \ddots & \vdots\\
				0 & 0 & \cdots & \lambda_n - \lambda\\
			\end{bmatrix}.
			$$
			
			Repeat for all $i$.
			
		\end{Proof}
	
\end{Theorem}





\begin{center}
	\vspace{.25in}
	\fadeline
	\vspace{.25in}
	
	\section{Inner Product Spaces}
	
	\vspace{.1in}
\end{center}



\begin{Def}
	
	Let $V$ be a vector space over $k = \mathbb{R}\ \textnormal{or}\ \mathbb{C}$. An \textbf{inner product} on $V$ is a function $\langle \cdot, \cdot \rangle : V \times V \longrightarrow \mathbb{C}$ such that
	
	\begin{enumerate}
		
		\item $\langle v, v \rangle \in \mathbb{R}^+$ for all nonzero $v \in V$ and $\langle v, v \rangle = 0$ if and only if $v = 0$.
		\item $\langle cu + v, w \rangle = c \langle u, w \rangle + \langle v, w \rangle$ for all $u, v, w \in V$ and $c \in k$.
		\item $\langle u, v \rangle = \overline{\langle v, u \rangle}$ for all $u, v \in V$.
		
	\end{enumerate}
	
	An \textbf{inner product space} is a vector space equipped with an inner product.
	
\end{Def}



\begin{Def}
	
	The \textbf{norm} of an element $v \in V$ is $||v|| = \sqrt{\langle v, v \rangle}$.
	
\end{Def}



\begin{Def}
	
	The \textbf{distance} between two vectors $u, v \in V$ is $||u - v||$.
	
\end{Def}



\begin{Proposition}
	
	For all $v \in V$ and $c \in k$, $||cv|| = |c| \cdot ||v||$.
	
	\begin{Proof}
		%
		Since $||cv||^2 = \langle cv, cv \rangle = c \overline{c} \langle v, v \rangle = |c|^2 ||v||^2$, $||cv|| = |c| \cdot ||v||$.
		
	\end{Proof}
	
\end{Proposition}



\begin{Def}
	
	Two vectors $u, v \in V$ are \textbf{orthogonal} if $\langle u, v \rangle = 0$.
	
\end{Def}



\begin{Proposition}
	
	\textbf{(The Pythagorean Theorem)} Let $u, v \in V$ be orthogonal. Then $||u + v||^2 = ||u||^2 + ||v||^2$.
	
	\begin{Proof}
		%
		We have $||u + v||^2 = \langle u+v, u+v \rangle = \langle u, u \rangle + \langle u, v \rangle + \langle v, u \rangle + \langle v, v \rangle = \langle u, u \rangle + \langle v, v \rangle = ||u||^2 + ||v||^2$.
		
	\end{Proof}
	
\end{Proposition}



\begin{Proposition}
	
	\textbf{(The Cauchy-Schwarz Inequality)} For all $u, v \in V$, $||u|| \cdot ||v|| \geq | \langle u, v \rangle |$.
	
	\begin{Proof}
		%
		Let $c = \frac{\langle u, v \rangle}{\langle v, v \rangle}$. Then $||u||^2 ||v||^2 = ||u - cv + cv||^2 ||v||^2$, and since $u - cv$ is orthogonal to $cv$, $||u||^2 ||v||^2 = \left( ||u - cv||^2 + ||cv||^2 \right) ||v||^2 \geq ||cv||^2 ||v||^2 = |c|^2 ||v||^4 = | \langle u, v \rangle |^2$.
		
	\end{Proof}
	
\end{Proposition}



\begin{Proposition}
	
	For all $u, v \in V$, $||u|| + ||v|| \geq ||u + v||$.
	
	\begin{Lemma}
		
		For all $z \in \mathbb{C}$, $2|z| \geq z + \overline{z}$.
		
		\begin{Proof}
			%
			If $z = a+bi$, then $2|z| = 2|a+bi| = 2\sqrt{a^2 + b^2} \geq 2\sqrt{a^2} = 2a = z + \overline{z}$.
			
		\end{Proof}
		
	\end{Lemma}
	
	\begin{Proof}
		%
		We have $\left( ||u|| + ||v|| \right)^2 = ||u||^2 + 2||u|| \cdot ||v|| + ||v||^2 \geq ||u||^2 + 2 | \langle u, v \rangle | + ||v||^2 \geq ||u||^2 + \langle u, v \rangle + \overline{\langle u, v \rangle} + ||v||^2 = \langle u, u \rangle + \langle u, v \rangle + \langle v, u \rangle + \langle v, v \rangle = \langle u + v, u + v \rangle = ||u + v||^2$.
		
	\end{Proof}

\end{Proposition}



\begin{Def}
	
	Vectors $e_1, ..., e_k \in V$ are \textbf{orthonormal} if $||e_i|| = 1$ for all $i$ and $\langle e_i, e_j \rangle = 0$ for all $i \neq j$.
	
\end{Def}



\begin{Proposition}
	
	If $e_1, ..., e_k \in V$ are orthonormal, then $||c_1 e_1 + \cdots + c_k e_k||^2 = |c_1|^2 + \cdots + |c_k|^2$.
	
	\begin{Proof}
		%
		We will induct upon k. The base case is obvious, since $||c_1 e_1||^2 = |c_1|^2 ||e_1||^2 = |c_1|^2$. For the induction step, assume $||c_1 e_1 + \cdots | c_k e_k||^2 = |c_1|^2 + \cdots + |c_k|^2$. Since $c_1 e_1 + \cdots + c_k e_k$ and $c_{k+1} e_{k+1}$ are orthogonal, $||c_1 e_1 + \cdots + c_{k+1} e_{k+1}||^2 = ||c_1 e_1 + \cdots + c_k e_k||^2 + ||c_{k+1} e_{k+1}||^2 = |c_1|^2 + \cdots + |c_k|^2 + |c_{k+1}|^2$.
		
	\end{Proof}
	
\end{Proposition}



\begin{Proposition}
	
	Orthonormal vectors are linearly independent.
	
	\begin{Proof}
		%
		Suppose $e_1, ..., e_k \in V$ are orthonormal and $c_1 e_1 + \cdots + c_k e_k = 0$. Then $||c_1 e_1 + \cdots + c_k e_k||^2 = |c_1|^2 + \cdots + |c_k|^2 = 0$, so $c_1 = \cdots = c_k = 0$.
		
	\end{Proof}
	
\end{Proposition}



\begin{Proposition}
	
	Let $\{e_1, ..., e_n\}$ be an orthonormal basis for $V$ and let $v \in V$. Then $v = \langle v, e_1 \rangle e_1 + \cdots + \langle v, e_n \rangle e_n$.
	
	\begin{Proof}
		%
		If $v = c_1 e_1 + \cdots + c_n e_n$, then $\langle v, e_i \rangle = \langle c_1 e_1 + \cdots + c_n e_n, e_i \rangle = \langle c_i e_i, e_i \rangle = c_i$.
		
	\end{Proof}
	
\end{Proposition}



\begin{Theorem}
	
	\textbf{(The Gram-Schmidt Process)} Every finite-dimensional inner product space has an orthonormal basis.
	
	\begin{Proof}
		%
		Let $\{v_1, ..., v_n\}$ be a basis for $V$. Let $e_1' = v_1$ and $e_1 = \frac{e_1'}{||e_1'||}$. Then for each $i \in \{2, ..., n\}$, let
		%
		$$
		e_i' = v_i - \left( \langle v_1, e_1 \rangle e_1 + \cdots + \langle v_{i-1}, e_{i-1} \rangle e_{i-1} \right)
		$$ 
		
		and $e_i = \frac{e_i'}{||e_i'||}$. Then $\{e_1, ..., e_n\}$ is an orthonormal basis for V.
		
	\end{Proof}
	
\end{Theorem}



\begin{Theorem}
	
	\textbf{(Riesz Representation)} Let $\varphi_u \in V'$ be defined by $\varphi_u v = \langle v, u \rangle$. Then for each $T \in V'$, there is a unique $u \in V$ such that $T = \varphi_u$.
	
	
	\begin{Proof}
		%
		Let $\{e_1, ..., e_n\}$ be an orthonormal basis for $V$ and let $u = \overline{Te_1}e_1 + \cdots + \overline{Te_n}e_n$. Then if $v = c_1 e_1 + \cdots + c_n e_n$,
		%
		\begin{align*}
			\varphi_u v &= \langle v, u \rangle\\
			&= \langle c_1 e_1 + \cdots + c_n e_n, \overline{Te_1}e_1 + \cdots + \overline{Te_n}e_n \rangle\\
			&= c_1 T e_1 + \cdots + c_n T e_n\\
			&= T(c_1 e_1 + \cdots + c_n e_n)\\
			&= Tv.
		\end{align*}
		
	\end{Proof}
	
\end{Theorem}



\begin{Def}
	
	Let $U \subseteq V$. The \textbf{orthogonal complement} to $U$ is the set $U^\perp = \{v \in V\ |\ \langle u, v \rangle = 0\ \textnormal{for all}\ u \in U\}$.
	
\end{Def}



\begin{Proposition}
	
	If $U$ is a subspace of $V$, then so is $U^\perp$.
	
\end{Proposition}



\begin{Theorem}
	
	If $U$ is a finite-dimensional subspace of $V$, then $V = U \oplus U^\perp$.
	
	\begin{Proof}
		%
		Let $\{e_1, ..., e_n\}$ be an orthonormal basis for $U$, let $v \in V$, and let $u = c_1 e_1 + \cdots + c_n e_n \in U$. Then $v = u + (v - u)$. If $v - u \in U^\perp$, this will be an expression of $v$ in $U + U^\perp$. For $v - u$ to be in $U^\perp$, $\langle v-u, e_i \rangle = 0$ for all $i$, so $c_i = \langle u, e_i \rangle = \langle v, e_i \rangle$ for all $i$. Thus $u$ is completely determined by $v$, so the expression of $v$ as $u + (v - u)$ is unique. Thus $V = U \oplus U^\perp$.
		
	\end{Proof}
	
	\begin{Corollary}
		
		If $U$ is a finite-dimensional subspace of $V$, then $\dim V = \dim U + \dim U^\perp$.
		
	\end{Corollary}
	
\end{Theorem}



\begin{Proposition}
	
	Let $U \subseteq V$. Then $(U^\perp)^\perp = U$.
	
	\begin{Proof}
		%
		Let $u \in U$ and $v \in U^\perp$. Then $\langle u, v \rangle = 0$ by definition, so $u \in (U^\perp)^\perp$. Thus $U \subseteq (U^\perp)^\perp$. Also, $\dim U + \dim U^\perp = \dim V = \dim U^\perp + \dim (U^\perp)^\perp$, so $\dim U = \dim (U^\perp)^\perp$. Thus $U = (U^\perp)^\perp$.
		
	\end{Proof}
	
\end{Proposition}



\begin{Def}
	
	The \textbf{projection} of $V$ onto a subspace $U$ is the linear map $P_U \in \mathcal{L}(V, U)$ given by $P_U v = u$, where $v = u + u' \in U \oplus U^\perp$.
	
\end{Def}



\begin{Proposition}
	
	Let $U$ be a subspace of a vector space $V$ with $\dim U = k$ and $\dim V = n$, and let $\{u_1, ..., u_k, u_{k+1}', ..., u_n'\}$ be a basis for $V$ composed of bases for $U$ and $U^\perp$. Then
	
	$$
	M(P_U) = \begin{bmatrix}
		1 & 0 & \cdots & 0 & 0 & \cdots & 0 & 0\\
		0 & 1 & \cdots & 0 & 0 & \cdots & 0 & 0\\
		\vdots & \vdots & \ddots & \vdots & \vdots & \ddots & \vdots & \vdots\\
		0 & 0 & \cdots & 1 & 0 & \cdots & 0 & 0\\
		0 & 0 & \cdots & 0 & 0 & \cdots & 0 & 0\\
		\vdots & \vdots & \ddots & \vdots & \vdots & \ddots & \vdots & \vdots\\
		0 & 0 & \cdots & 0 & 0 & \cdots & 0 & 0\\
		0 & 0 & \cdots & 0 & 0 & \cdots & 0 & 0
	\end{bmatrix}.
	$$
	
\end{Proposition}



\begin{Theorem}
	
	Let $V$ be a vector space and $U$ a subspace. Then for all $v \in V$ and $u \in U$, $||v - P_U v|| \leq ||v - u||$; that is, the closest vector to $v$ in $U$ is $P_U v$.
	
	\begin{Proof}
		%
		Since $v - P_U v \notin U$, $v - P_U v \in U^\perp$, so $v - P_U v$ and $P_U v - u$ are orthogonal. Then $||v - u||^2 = ||v - P_U v + P_U v - u||^2 = ||v - P_U v||^2 + ||P_U v - u||^2 \geq ||v - P_U v||^2$.
		
	\end{Proof}
	
\end{Theorem}





\begin{center}
	\vspace{.25in}
	\fadeline
	\vspace{.25in}
	
	\section{Linear Maps and Inner Products}
	
	\vspace{.1in}
\end{center}



\begin{Def}
	
	Let $T \in \mathcal{L}(V, W)$. The \textbf{adjoint} of $T$ is the linear map $T^* \in \mathcal{L}(W, V)$ such that $\langle Tv, w \rangle_W = \langle v, T^*w \rangle_V$ for all $v \in V$ and $w \in W$.
	
\end{Def}



\begin{Proposition}
	
	Let $T \in \mathcal{L}(U, V)$, $S \in \mathcal{L}(V, W)$, and $c \in k$. Then
	
	\begin{enumerate}
		
		\item $(cT + S)^* = \overline{c} T^* + S^*$.
		\item $(T^*)^* = T$.
		\item $I^* = I$.
		\item $(ST)^* = T^* S^*$.
		
	\end{enumerate}
	
\end{Proposition}



\begin{Theorem}
	
	Let $\{e_1, ..., e_n\}$ and $\{f_1, ..., f_m\}$ be orthonormal bases for $V$ and $W$ and let $T \in \mathcal{L}(V, W)$. Then $M(T^*) = \overline{M(T)}^\textnormal{T}$.
	
	\begin{Proof}
		%
		The $j$th column of $M(T^*)$ is $T^* f_j$ expressed in the basis $\{e_1, ..., e_n\}$. Since this is orthonormal, $T^* f_j = \langle T^* f_j, e_1 \rangle e_1 + \cdots + \langle T^* f_j, e_n \rangle e_n$, so $M(T^*)_{ij} = \langle T^* f_j, e_i$. But $M(T)_{ji} = \langle Te_i, f_j \rangle = \langle e_i, T^* f_j \rangle = \overline{\langle T^* f_j, e_i \rangle} = \overline{M(T^*)_{ij}}$, so $M(T^*) = \overline{M(T)}^\textnormal{T}.$
		
	\end{Proof}
	
\end{Theorem}



\begin{Def}
	
	A linear map $T \in \mathcal{L}(V)$ is \textbf{self-adjoint} if $T^* = T$.
	
\end{Def}



\begin{Proposition}
	
	Let $T \in \mathcal{L}(V)$ be self-adjoint. Then if $\langle Tv, v \rangle = 0$ for all $v \in V$, $T = 0$.
	
\end{Proposition}



\begin{Def}
	
	A linear map $T \in \mathcal{L}(V)$ is \textbf{normal} if $T^* T = TT^*$.
	
\end{Def}



\begin{Proposition}
	
	A linear map $T \in \mathcal{L}(V)$ is \textbf{normal} if and only for all $v \in V$, $||Tv|| = ||T^* v||$.
	
	\begin{Proof}
		%
		$(\Rightarrow)$ If $T$ is normal, then $\langle Tv, Tv \rangle = \langle v, T^* T v \rangle = \langle v, T T^* v \rangle = \langle T^* v, T^* v \rangle$.\\
		
		$(\Leftarrow)$ Suppose $\langle Tv, Tv \rangle = \langle T^* v, T^* v \rangle$ for all $v \in V$. Then $\langle TT^* v - T^* T v, v \rangle = 0$ for all $v \in V$, so $TT^* - T^* T = 0$.
		
	\end{Proof}
	
\end{Proposition}



\begin{Proposition}
	
	If $T$ is normal and $Tv = \lambda v$ for some $v \neq 0$, then $T^* v = \overline{\lambda} v$.
	
	\begin{Proof}
		%
		$(T- \lambda I)^* (T - \lambda I)$ is normal, since $(T- \lambda I)^* (T - \lambda I) = T^* T - \overline{\lambda} IT - \lambda IT + \lambda \overline{\lambda} I = (T- \lambda I)(T - \lambda I)^*$. Then $0 = ||(T - \lambda I)v|| = ||(T - \lambda I)^* v|| = ||(T^* - \overline{\lambda})v||$, so $T^* v = \overline{\lambda} v$.
		
	\end{Proof}
	
\end{Proposition}



\begin{Proposition}
	
	Let $T \in \mathcal{L}(V)$ be normal. If $v$ and $w$ are eigenvectors of $T$ with distinct eigenvalues $\lambda_1$ and $\lambda_2$, then $v$ and $w$ are orthogonal.
	
	\begin{Proof}
		%
		Since $\lambda_1 \neq \lambda_2$, $\lambda_1 - \lambda_2 \neq 0$. Then $(\lambda_1 - \lambda_2) \langle v, w \rangle = \langle \lambda_1 v - \lambda_2 v, w \rangle = \langle Tv, w \rangle - \langle v, \overline{\lambda_2} w \rangle = \langle Tv, w \rangle - \langle v, T^* w \rangle = \langle Tv, w \rangle - \langle Tv, w \rangle = 0$, so $\langle v, w \rangle = 0$.
		
	\end{Proof}
	
\end{Proposition}



\begin{Theorem}
	
	\textbf{(Complex Spectral)} Let $V$ be a finite-dimensional vector space over $\mathbb{C}$ and let $T \in \mathcal{L}(V)$. Then $T$ is normal if and only if there is an orthonormal basis of eigenvectors of $T$ for $V$.
	
	\begin{Proof}
		%
		$(\Rightarrow)$ We will induct on $n = \dim V$. The base case is trivial, since if $\dim V = 1$ and $T \in \mathcal{L}(V)$, then any nonzero unit vector in $V$ constitutes an orthonormal basis of eigenvectors of $T$.\\
		
		Suppose the theorem holds for $n - 1$-dimensional vector spaces and let $T \in \mathcal{L}(V)$ be normal with $\dim V = n$. Let $\{e_1, ..., e_n\}$ be an orthonormal basis for $V$ such that $M(T)$ is upper triangular (this is possible, since the Gram-Schmidt process preserves upper triangularity). Then we have
		
		$$
		M(T) = \begin{bmatrix}
		
		\lambda_1 & *_{1,2} & \cdots & *_{1,n}\\
		0 & \lambda_2 & \cdots & *_{2,n}\\
		\vdots & \vdots & \ddots & \vdots\\
		0 & 0 & \cdots & \lambda_n 
		
		\end{bmatrix},
		$$
		
		where $*_i$ is the vector of the first $i - 1$ entries in column $i$ of $M(T)$. Consider the first column of $M(T)$ and $M(T^*) = \overline{M(T)}^\textnormal{T}$. We have $Te_1 = \lambda_1 e_1$ and $T^* e_1 = \overline{\lambda_1} e_1 + \overline{*_{1,2}} e_2 + \cdots + \overline{*_{1,n}} e_n$, but $T$ is normal, so $||Te_1|| = ||T^* e_1||$. Thus $|\lambda_1|^2 = |\overline{\lambda_1}|^2 + |*_{1,2}|^2 + \cdots + |*_{1,n}|^2$, so $|*_{1,2}|^2 + \cdots + |*_{1,n}|^2 = 0$, and therefore $*_{1,2} = \cdots + *_{1,n} = 0$. Thus the first row and column of $M(T)$ are zero, except for $\lambda_1$, and similarly for $M(T^*)$. By restricting $T$ to $\Span \{e_2, ..., e_n\}$, which has dimension $n - 1$, we are done by induction.\\
		
		$(\Leftarrow)$ If $\{e_1, ..., e_n\}$is an orthonormal basis of eigenvectors of $T$, then
		
		$$
		M(T)M(T^*) = \begin{bmatrix}
		
		\lambda_1 & \cdots & 0\\
		\vdots & \ddots & \vdots\\
		0 & \cdots & \lambda_n
		
		\end{bmatrix} \begin{bmatrix}
		
		\overline{\lambda_1} & \cdots & 0\\
		\vdots & \ddots & \vdots\\
		0 & \cdots & \overline{\lambda_n} 
		
		\end{bmatrix} = \begin{bmatrix}
		
		\overline{\lambda_1} & \cdots & 0\\
		\vdots & \ddots & \vdots\\
		0 & \cdots & \overline{\lambda_n} 
		
		\end{bmatrix} \begin{bmatrix}
		
		\lambda_1 & \cdots & 0\\
		\vdots & \ddots & \vdots\\
		0 & \cdots & \lambda_n
		
		\end{bmatrix} = M(T^*) M(T),
		$$
		
		so $T$ is normal.
		
	\end{Proof}
	
\end{Theorem}



\begin{Theorem}
	
	\textbf{(Real Spectral)} Let $V$ be a finite-dimensional vector space over $\mathbb{R}$ and let $T \in \mathcal{L}(V)$. Then $T$ is self-adjoint if and only if there is an orthonormal basis of eigenvectors of $T$ with real eigenvalues.
	
	\begin{Proof}
		%
		By the Complex Spectral Theorem, there is an orthonormal basis for $V$ of eigenvectors of $T$. Since $T$ is self-adjoint, $M(T) = M(T^*) = \overline{M(T)}^\textnormal{T} = \overline{M(T)}$. Thus each $\lambda_i = \overline{\lambda_i}$, so all of $T$'s eigenvalues are real.
		
	\end{Proof}

\end{Theorem}



\begin{Def}
	
	A linear map $T \in \mathcal{L}(V)$ is \textbf{positive} if $T$ is self-adjoint and $\langle Tv, v \rangle \geq 0$ for all $v \in V$.
	
\end{Def}



\begin{Proposition}
	
	Let $T \in \mathcal{L}(V)$ be self-adjoint. Then $T$ is positive if and only if every eigenvalue of $T$ is nonnegative.
	
	\begin{Proof}
		%
		Let $\{e_1, ..., e_n\}$ be an orthonormal basis for $V$ of eigenvectors of $T$ with eigenvectors $\lambda_1, ..., \lambda_n$. Then $T$ is positive if and only if $\langle T(c_1 e_1 + \cdots + c_n e_n), c_1 e_1 + \cdots + c_n e_n \rangle \geq 0$ for all $c_1, ..., c_n \in k$, if and only if $\langle c_1 \lambda_1 e_1 + \cdots + c_n \lambda_n  e_n, c_1 e_1 + \cdots + c_n e_n \rangle \geq 0$ for all $c_1, ..., c_n \in k$, if and only if $|c_1|^2 \lambda_1 + \cdots |c_n|^2 \lambda_n \geq 0$ for all $c_1, ..., c_n \in k$, if and only if each $\lambda_i \geq 0$ (for each $i$, choose $c_i = 1$ and $c_j = 0$ for $j \neq i$).
		
	\end{Proof}
	
\end{Proposition}



\begin{Def}
	
	Let $T \in \mathcal{L}(V)$. A \textbf{square root} of $T$ is a linear map $R \in \mathcal{L}(V)$ such that $R^2 = T$.
	
\end{Def}



\begin{Theorem}
	
	Let $T \in \mathcal{L}(V)$ be positive. Then there is a unique positive square root of $T$.
	
	\begin{Proof}
		%
		We will only show existence --- the proof of uniqueness is difficult, tedious, and unenlightening. Let $\{e_1, ..., e_n\}$ be an orthonormal basis for $V$ of eigenvectors of $T$ with eigenvectors $\lambda_1, ..., \lambda_n$. Then each $\lambda_i \geq 0$, so the map $R \in \mathcal{L}(V)$ defined by $Re_i = \sqrt{\lambda_i} e_i$ is positive, and clearly $R^2 = T$. Thus $T$ has a positive square root.
				
	\end{Proof}
	
\end{Theorem}



\begin{Def}
	
	Let $T \in \mathcal{L}(V)$ be positive. The unique positive square root of $T$ is denoted $\sqrt{T}$.
	
\end{Def}



\begin{Def}
	
	A linear map $T \in \mathcal{L}(V)$ is an \textbf{isometry} if $||Tv|| = ||v||$ for all $v \in V$.
	
\end{Def}



\begin{Proposition}
	
	A linear map $T \in \mathcal{L}(V)$ is an isometry if and only if $T^* T = I$.
	
	\begin{Proof}
		%
		$T$ is an isometry if and only if $\langle Tv, Tv \rangle = \langle v, v \rangle$ for all $v \in V$, if and only if $\langle T^* T v, v \rangle - \langle Iv, v \rangle = 0$ for all $v \in V$, if and only if $T^* T - I = 0$, since $T^* T - I$ is self-adjoint.
		
	\end{Proof}
	
\end{Proposition}



\begin{Theorem}
	
	A linear map $T \in \mathcal{L}(V)$ is an isometry if and only if there is an orthonormal basis of eigenvectors of $T$ with eigenvalues $\lambda_1, ..., \lambda_n$ such that $|\lambda_i| = 1$.
	
	\begin{Proof}
		%
		$(\Rightarrow)$ If $T$ is an isometry, then it is normal, so there is an orthonormal basis of eigenvectors $\{e_1, ..., e_n\}$ with eigenvalues $\lambda_1, ..., \lambda_n$ by the Complex Spectral Theorem. Then $|\lambda_i| = ||\lambda_i e_i|| = ||Te_i|| = ||e_i|| = 1$.\\
		
		$(\Leftarrow)$ Let $v = c_1 e_1 + \cdots + c_n e_n \in V$. Then $||Tv|| = ||c_1 \lambda_1 e_1 + \cdots + c_n \lambda_n e_n|| = ||c_1 e_1 + \cdots + c_n e_n|| = ||v||$, so $T$ is an isometry.
		
	\end{Proof}
	
\end{Theorem}



\begin{Theorem}
	
	\textbf{(Polar Decomposition)} Let $T \in \mathcal{L}(V)$. Then there is an isometry $S \in \mathcal{L}(V)$ such that $T = S\sqrt{T^* T}$.
	
	\begin{Proof}
		%
		Let $\{e_1, ..., e_n\}$ be an orthonormal basis of eigenvectors of $T*T$ with eigenvalues $\lambda_1, ..., \lambda_n$ and suppose without loss of generality that $\lambda_1 = \cdots = \lambda_k = 0$. Let $\{f_1, ..., f_k\}$ be an orthonormal basis for $(\Range T)^\perp$ (the dimension is $k$ since $\dim \Range T = \dim \Null T^*$). Then define $S$ by
		
		$$
		Se_i = \begin{cases} 
			f_i, & i \leq k \\
			\frac{1}{\sqrt{\lambda_i}}Te_i, & i > k
		\end{cases}.
		$$
		
		It follows that $S$ is an isometry and $T = S \sqrt{T^* T}$.
		
	\end{Proof}
	
\end{Theorem}



\begin{Def}
	
	The \textbf{singular values} of a linear map $T \in \mathcal{L}(V, W)$ are $\sigma_1, ..., \sigma_k$, where $\sigma_i = \sqrt{\lambda_i}$ and $\lambda_1, ..., \lambda_k$ are the nonzero eigenvalues of $T^* T$.
	
\end{Def}



\begin{Theorem}
	
	\textbf{(Singular Value Decomposition)} Let $V$ and $W$ be vector spaces with $\dim V = n$ and $\dim W = m$. Let $\{e_1, ..., e_n\}$ be an orthonormal basis of eigenvectors of $T^* T$ with eigenvalues $\lambda_1 \geq \cdots \geq \lambda_k > 0 = \lambda_{k+1} = \cdots = \lambda_n$. Then there is an orthonormal basis $\{f_1, ..., f_m\}$ for $W$ such that
	
	$$Tv = \sigma_1 \langle v, e_1 \rangle f_1 + \cdots + \sigma_k \langle v, e_k \rangle f_k,$$
	
	or equivalently,
	
	$$
	M(T) = \begin{bmatrix}
	
	\sigma_1 & 0 & \cdots & 0 & 0 & \cdots & 0\\
	0 & \sigma_2 & \cdots & 0 & 0 & \cdots & 0\\
	\vdots & \vdots & \ddots & \vdots & \vdots & \ddots & \vdots\\
	0 & 0 & \cdots & \sigma_k & 0 & \cdots & 0\\
	0 & 0 & \cdots & 0 & 0 & \cdots & 0\\
	\vdots & \vdots & \ddots & \vdots & \vdots & \ddots & \vdots\\
	0 & 0 & \cdots & 0 & 0 & \cdots & 0
	
	\end{bmatrix}.
	$$
	
	\begin{Proof}
		%
		Let $f_i = \frac{1}{\sigma_i}Te_i$ for all $i \leq k$ and extend and orthonormalize to form a basis for $W$.
		
	\end{Proof}
	
\end{Theorem}



\begin{Theorem}
	
	Let $A \in M_{m \times n}(k)$. Then there are isometries $U \in M_m(k)$ and $V \in M_n(k)$ such that $A = U \Sigma V^*$, where $\Sigma \in M_{m, n}$ contains the singular values of $A$.
	
	\begin{Proof}
		%
		$$\textnormal{Let}\ U = \begin{bmatrix}
		
		| & & |\\
		f_1 & \cdots & f_m\\
		| & & |
		
		\end{bmatrix},\ \Sigma = \begin{bmatrix}
		
		\sigma_1 & \cdots & 0 & \cdots & 0\\
		\vdots & \ddots & \vdots & \ddots & \vdots\\
		0 & \cdots & \sigma_k & \cdots & 0\\
		\vdots & \ddots & \vdots & \ddots & \vdots\\
		0 & \cdots & 0 & \cdots & 0
		
		\end{bmatrix},\ \textnormal{and}\ V = \begin{bmatrix}
		
		| & & |\\
		e_1 & \cdots & e_n\\
		| & & |
		
		\end{bmatrix}.$$
		
	\end{Proof}
	
\end{Theorem}



\begin{Def}
	
	$$\textnormal{Let}\ \Sigma = \begin{bmatrix}
	
	\sigma_1 & \cdots & 0 & \cdots & 0\\
	\vdots & \ddots & \vdots & \ddots & \vdots\\
	0 & \cdots & \sigma_k & \cdots & 0\\
	\vdots & \ddots & \vdots & \ddots & \vdots\\
	0 & \cdots & 0 & \cdots & 0
	
	\end{bmatrix}.\ \textnormal{The \textbf{pseudoinverse} to}\ \Sigma\ \textnormal{is}\ \Sigma^+ = \begin{bmatrix}
	
	\sigma_1^{-1} & \cdots & 0 & \cdots & 0\\
	\vdots & \ddots & \vdots & \ddots & \vdots\\
	0 & \cdots & \sigma_k^{-1} & \cdots & 0\\
	\vdots & \ddots & \vdots & \ddots & \vdots\\
	0 & \cdots & 0 & \cdots & 0
	
	\end{bmatrix}^\textnormal{T}.
	$$
	
\end{Def}



\begin{Def}
	
	Let $A = U \Sigma V^*$. The \textbf{pseudoinverse} to $A$ is $A^+ = V \Sigma^+ U^*$.
	
\end{Def}



\begin{Proposition}
	
	Let $A \in M_n(k)$ be invertible. Then $A^+ = A^{-1}$.
	
	\begin{Proof}
		%
		Since $A$ is invertible, no entry along the diagonal of $\Sigma$ is zero, so $\Sigma^+ = \Sigma^{-1}$. Since $U$ and $V$ are isometries, $V^* V = U U^* = I$, so $AA^+ = U \Sigma V^* V \Sigma^+ U^* = U \Sigma \Sigma^+ U^* = U U^* = I$. Thus $A^+ = A^{-1}$.
		
	\end{Proof}
	
\end{Proposition}



\begin{Theorem}
	
	Let $A \in M_{m, n}(k)$. Then the map given by $AA^+$ is the projection onto $\Range A$, so the vector $\mathbf{x}$ closest to a solution to $A\mathbf{x} = \mathbf{b}$ is $\mathbf{x} = A^+ \mathbf{b}$.
	
	\begin{Proof}
		%
		Let $\{e_1, ..., e_n\}$ be an orthonormal basis of eigenvectors of $A^* A$, let $\sigma_1, ..., \sigma_k$ be the singular values of $A$, and let $\{f_1, ..., f_m\}$ be the orthonormal basis given by the Singular Value Decomposition of $A$. Then
		
		\begin{align*}
			AA^+ v &= A(\frac{1}{\sigma_1} \langle v, f_1 \rangle e_1 + \cdots + \frac{1}{\sigma_k} \langle v, f_k \rangle e_k)\\
			&= \sigma_1 \langle \frac{1}{\sigma_1} \langle v, f_1 \rangle e_1, e_1 \rangle f_1 + \cdots + \sigma_k \langle \frac{1}{\sigma_k} \langle v, f_k \rangle e_k, e_k \rangle f_k\\
			&= \langle v, f_1 \rangle f_1 + \cdots + \langle v, f_k \rangle f_k,
		\end{align*}
		
		so if $v = c_1 f_1 + \cdots + c_m f_m$, then $AA^+ v = c_1 f_1 + \cdots + c_k f_k$. Since $\Range A = \Span \{f_1, ..., f_k\}$, $AA^+ = P_{\Range A}$.
		
	\end{Proof}
	
\end{Theorem}



\begin{Def}
	
	$$\textnormal{Let}\ \Sigma = \begin{bmatrix}
	
	\sigma_1 & \cdots & 0 & \cdots & 0\\
	\vdots & \ddots & \vdots & \ddots & \vdots\\
	0 & \cdots & \sigma_k & \cdots & 0\\
	\vdots & \ddots & \vdots & \ddots & \vdots\\
	0 & \cdots & 0 & \cdots & 0
	
	\end{bmatrix}.\ \textnormal{The \textbf{rank \emph{r} approximation} to}\ \Sigma\ \textnormal{is}\ \Sigma_r = \begin{bmatrix}
	
	\sigma_1 & \cdots & 0 & \cdots & 0\\
	\vdots & \ddots & \vdots & \ddots & \vdots\\
	0 & \cdots & \sigma_r & \cdots & 0\\
	\vdots & \ddots & \vdots & \ddots & \vdots\\
	0 & \cdots & 0 & \cdots & 0
	
	\end{bmatrix}
	$$
	
\end{Def}



\begin{Def}
	
	Let $A = U \Sigma V^*$. The \textbf{rank \emph{r} approximation} to $A$ is $A_r = U \Sigma_r V^*$.
	
\end{Def}



\begin{Theorem}
	
	Let $A \in M_{m, n}(k)$. Then $A_r$ is the rank r matrix closest to $A$ --- that is, it minimizes $||A - X||$, where $\langle A, X \rangle = \textnormal{trace}\ (X^* A)$.
	
\end{Theorem}





\begin{center}
	\vspace{.25in}
	\fadeline
	\vspace{.25in}
	
	\section{Determinants}
	
	\vspace{.1in}
\end{center}



\begin{Def}
	
	The \textbf{symmetric group} $S_n$ is the group $\{\sigma : \{1, ..., n\} \hookrightarrow\mathrel{\mspace{-15mu}}\rightarrow \{1, ..., n\} \}$, with composition given by composition of functions. The elements of $S_n$ are called \textbf{permutations} and are written as $\sigma = \sigma_1 \cdots \sigma_n$, where $\sigma(i) = \sigma_i$.
	
\end{Def}



\begin{Def}
	
	Let $\sigma$ be a permutation. The \textbf{inversion} of $\sigma$, denoted $\textnormal{inv }\sigma$, is the number of pairs $(i, j)$ with $i < j$ and $\sigma_i > \sigma_j.$
	
\end{Def}



\begin{Def}
	
	The \textbf{sign} of a permutation $\sigma$ is $\textnormal{sign }\sigma = (-1)^{\textnormal{inv }\sigma}$.
	
\end{Def}



\begin{Proposition}
	
	Let $\sigma$ be a permutation and $\hat{\sigma}$ be a permutation identical to $\sigma$, except with $\sigma_i$ and $\sigma_j$ interchanged. Then $\textnormal{sign } \hat{\sigma} = -\textnormal{sign } \sigma$.
	
	\begin{Proof}
		%
		Suppose $\sigma = \textnormal{--- } i \textnormal{ --- } j \textnormal{ ---}$. Then $\hat{\sigma} = \textnormal{--- } j \textnormal{ --- } i \textnormal{ ---}$. Since any inversion that does not involve either $i$ or $j$ is unchanged from $\sigma$ to $\hat{\sigma}$, we need only consider those do. Any inversion of the form $(x, i)$ or $(j, x)$ is unchanged, since if $x < i$, then $x < j$, and if $x > j$, then $x > i$. Thus we only need to consider the $x$ that lie between $i$ and $j$. Each one causes two inversions in $\hat{\sigma}$ --- $(j, x)$ and $(x, i)$ --- and therefore does not affect $\textnormal{sign } \hat{\sigma}$. But we have not accounted for the inversion $(j, i)$. Thus $\textnormal{sign } \hat{\sigma} = -\textnormal{sign } \sigma$.
		
	\end{Proof}
	
\end{Proposition}



\begin{Def}
	
	Let A = $[a_{ij}] \in M_{m, n}(k)$. The \textbf{determinant} of $A$ is given by
	
	$$\det A = \scaleto{\sum\limits_{\sigma \in S_n}}{7ex} (\textnormal{sign } \sigma)(a_{\sigma_1, 1}) \cdots (a_{\sigma_n, n}).$$
	
\end{Def}



\begin{Theorem}
	
	The set
	
	$$\mathcal{A} = \{f : (\mathbb{R}^n)^n \longrightarrow \mathbb{R}\ |\ f(a_1, ..., a_i, ..., a_j, ..., a_n) = -f(a_1, ..., a_j, ..., a_i, ..., a_n),\ f \textnormal{ is coordinate-wise linear} \}$$
	
	has dimension $1$ (and therefore, one basis is $\{\det\}$).
	
	\begin{Lemma}
		
		Let $\{v_1, ..., v_n\}$ be a basis for $\mathbb{R}^n$ and let $f \in \mathcal{A}$. Then if $f(v_1, ..., v_n) = 0$, $f(w_1, ..., w_n) = 0$ for all $w_1, ...,w_n \in \mathbb{R}^n$.
		
		\begin{Proof}
			%
			Expand each $w_i$ as $w_i = c_{i1} v_1 + \cdots + c_{in} v_n$. Since $f$ is linear in each coordinate, we have
			
			$$f(w_1, ..., w_n) = f(c_{11} v_1 + \cdots + c_{1n} v_n, ..., c_{n1} v_1 + \cdots + c_{nn} v_n) = \scaleto{\sum}{4ex} c_i f(v_{i_1}, ..., v_{i_n}).$$
			
			Now any term of this sum with some $v_{i_j} = v_{i_k}$ will have $f(v_{i_1}, ..., v_{i_n}) = -f(v_{i_1}, ..., v_{i_n}) = 0$ by the previous result, and for the rest, we can rearrange the terms to get $f(v_{i_1}, ..., v_{i_n}) = \pm (v_1, ..., v_n) = 0$. Thus $f(w_1, ..., w_n) = 0$.
			
		\end{Proof}
		
	\end{Lemma}
	
	\begin{Proof}
		%
		Let $f, g \in \mathcal{A}$ with $g \neq 0$, let $\{v_1, ..., v_n\}$ be a basis for $\mathbb{R}^n$, and let $c = \frac{f(v_1, ..., v_n)}{g(v_1, ..., v_n)}$ ($g(v_1, ..., v_n) \neq 0$, since otherwise $g = 0$ by the lemma). Then $(f - cg)(v_1, ..., v_n = 0)$, so by the lemma, $f - cg = 0$. Thus $f = cg$, so every function in $\mathcal{A}$ is a multiple of another.
		
	\end{Proof}
	
\end{Theorem}



\begin{Theorem}
	
	Let $A, B \in M_{n}(k)$. Then $\det AB = (\det A)(\det B)$.
	
	\begin{Proof}
		%
		Define $f \in \mathcal{A}$ by $f(C) = \det AC$. By the previous result, there is a $c$ such that $f = c \cdot \det$. Since $f(I) = \det A$ and $f(I) = c \cdot \det I = c$, $c = \det A$. Then $f(B) = \det AB = c \cdot \det B = (\det A)(\det B)$.
		
	\end{Proof}
	
\end{Theorem}



\begin{Theorem}
	
	Let $A \in M_n(k)$ with eigenvalues $\lambda_1, ..., \lambda_n$. Then $\det A = \lambda_1 \cdots \lambda_n$.
	
	\begin{Proof}
		%
		Let $\{v_1, ..., v_n\}$ be a basis for $\mathbb{R}^n$ such that $A$ is upper triangular. Then $A = SUS^{-1}$, where
		
		$$S = \begin{bmatrix}
		
		| & & |\\
		v_1 & \cdots & v_n\\
		| & & |
		
		\end{bmatrix} \textnormal{ and } U = \begin{bmatrix}
		
		\lambda_1 & \cdots & *\\
		\vdots & \ddots & \vdots\\
		0 & \cdots & \lambda_n
		
		\end{bmatrix}.$$
		
		Then $\det A = \det SUS^{-1} = (\det S)(\det U)(\det S^{-1}) = \det U = \lambda_1 \cdots \lambda_n$.
		
	\end{Proof}
	
\end{Theorem}



\begin{center}
	\vspace{.25in}
	\fadeline
	\vspace{.25in}
	
	\href{http://www.cruzgodar.com/notes/source/linear-algebra.tex}{Source}
\end{center}





\end{document}