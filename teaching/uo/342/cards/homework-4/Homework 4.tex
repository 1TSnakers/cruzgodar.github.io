\documentclass{article}
\usepackage[utf8]{inputenc}
\usepackage[T1]{fontenc}
\usepackage{amsmath}
\usepackage{amsfonts}
\usepackage{amssymb}
\usepackage[dvipsnames]{xcolor}
\usepackage{enumitem}
\usepackage{titlesec}
\usepackage{graphicx}
\usepackage[total={6.5in, 9in}, heightrounded]{geometry}
\usepackage{hyperref}

\graphicspath{{graphics/}}
\setenumerate[0]{label=\alph*)}
\setlength{\parindent}{0pt}
\setlength{\parskip}{8pt}
\setlength\fboxsep{0pt}
\renewcommand{\baselinestretch}{1.6}
\titleformat{\section}
{\normalfont \Large \bfseries \centering}{}{0pt}{}

\newcommand{\s}[1]{{\color{violet} #1}}

\begin{document}

\Large Name: \rule{2in}{0.15mm} \hfill Homework 4 | Math 341 | Cruz Godar \vspace{4pt} \normalsize

\textit{Due Wednesday of Week 5 at the start of class}

Complete the following problems and submit them as a pdf to Canvas. 8 points are awarded for thoroughly attempting every problem, and I'll select three problems to grade on correctness for 4 points each. Enough work should be shown that there is no question about the mathematical process used to obtain your answers.

\section{Section 4}

In problems 1--3, compute $\vec{v} \bullet \vec{w}$, $\left| \left| \vec{v} \right| \right|$, $\left| \left| \vec{w} \right| \right|$, the distance between $\vec{v}$ and $\vec{w}$, and the angle between them (in radians).

1. $\vec{v} = \left[\begin{array}{c} 1 \\ 2 \end{array}\right]$ and $\vec{w} = \left[\begin{array}{c} -3 \\ 2 \end{array}\right]$.

2. $\vec{v} = \left[\begin{array}{c} -7 \\ 2 \\ 0 \end{array}\right]$ and $\vec{w} = \left[\begin{array}{c} -1 \\ 1 \\ 1 \end{array}\right]$.

3. $\vec{v} = \left[\begin{array}{c} 1 \\ 2 \\ 4 \\ 3 \end{array}\right]$ and $\vec{w} = \left[\begin{array}{c} 2 \\ -1 \\ 3 \\ -4 \end{array}\right]$.

~\\

4. Let $X = \operatorname{span} \left\{ \left[\begin{array}{c} 1 \\ 1 \\ 1 \end{array}\right] \right\}$. Find a basis for $X^\perp$.

5. Let $X = \operatorname{span} \left\{ \left[\begin{array}{c} 1 \\ 2 \\ 1 \\ 1 \end{array}\right], \left[\begin{array}{c} 2 \\ 3 \\ 0 \\ -1 \end{array}\right] \right\}$. Find a basis for $X^\perp$.

6. Let $X$ be a subspace of $\mathbb{R}^n$. Show that $\dim X + \dim X^\perp = n$ by constructing a linear map whose kernel is exactly $X^\perp$. You may find it useful to recall that a matrix's image has dimension equal to the number of linearly independent rows.

~\\

In problems 7--9, show that the set of vectors is orthogonal and then normalize them all to produce an orthonormal basis. Then express the given vector $\vec{v}$ in that basis.

7. $\vec{v_1} = \left[\begin{array}{c} 1 \\ 2 \end{array}\right]$, $\vec{v_2} = \left[\begin{array}{c} -6 \\ 3 \end{array}\right]$, and $\vec{v} = \left[\begin{array}{c} 4 \\ 0 \end{array}\right]$.

8. $\vec{v_1} = \left[\begin{array}{c} 2 \\ 3 \\ 1 \end{array}\right]$, $\vec{v_2} = \left[\begin{array}{c} -1 \\ 2 \\ -4 \end{array}\right]$, $\vec{v_3} = \left[\begin{array}{c} -2 \\ 1 \\ 1 \end{array}\right]$, and $\vec{v} = \left[\begin{array}{c} 1 \\ 1 \\ 2 \end{array}\right]$.

9. $\vec{v_1} = \left[\begin{array}{c} 1 \\ 2 \\ 0 \\ -1 \end{array}\right]$, $\vec{v_2} = \left[\begin{array}{c} 1 \\ 2 \\ -6 \\ 5 \end{array}\right]$, $\vec{v_3} = \left[\begin{array}{c} -2 \\ 7 \\ 12 \\ 12 \end{array}\right]$, $\vec{v_4} = \left[\begin{array}{c} 5 \\ -2 \\ 1 \\ 1 \end{array}\right]$, and $\vec{v} = \left[\begin{array}{c} 2 \\ 1 \\ 2 \\ 1 \end{array}\right]$.

~\\

10. Let $A$ and $B$ be $n \times n$ unitary matrices. Is $AB$ always unitary? If so, explain why, and if not, give a brief counterexample.

11. Let $A$ be a matrix whose columns are orthogonal (but not necessarily orthonormal). Does $A$ still preserve lengths? If so, explain why, and if not, give a brief counterexample.

12. Continuing with the idea from the previous problem, let $A$ be a matrix whose columns are orthogonal, but not necessarily orthonormal. Is $A$ still necessarily invertible? If so, explain why, and if not, give a brief counterexample.


\end{document}