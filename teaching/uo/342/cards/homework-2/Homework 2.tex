\documentclass{article}
\usepackage[utf8]{inputenc}
\usepackage[T1]{fontenc}
\usepackage{amsmath}
\usepackage{amsfonts}
\usepackage{amssymb}
\usepackage[dvipsnames]{xcolor}
\usepackage{enumitem}
\usepackage{titlesec}
\usepackage{graphicx}
\usepackage[total={6.5in, 9in}, heightrounded]{geometry}
\usepackage{hyperref}

\graphicspath{{graphics/}}
\setenumerate[0]{label=\alph*)}
\setlength{\parindent}{0pt}
\setlength{\parskip}{8pt}
\setlength\fboxsep{0pt}
\renewcommand{\baselinestretch}{1.6}
\titleformat{\section}
{\normalfont \Large \bfseries \centering}{}{0pt}{}

\newcommand{\s}[1]{{\color{violet} #1}}

\begin{document}

\Large Name: \rule{2in}{0.15mm} \hfill Homework 2 | Math 342 | Cruz Godar \vspace{4pt} \normalsize

\textit{Due Wednesday of Week 3 at the start of class}

Complete the following problems and submit them as a pdf to Canvas. 8 points are awarded for thoroughly attempting every problem, and I'll select three problems to grade on correctness for 4 points each. Enough work should be shown that there is no question about the mathematical process used to obtain your answers.

\section{Section 1}

1. Suppose every 5 years, an average of 5\% of people from California move to Oregon, and 10\% move to Washington; 15\% from Oregon move to California and 10\% to Washington; and 5\% from Washington move to Oregon and 10\% to California. California currently has a population of 40 million people, Oregon has a population of 4 million, and Washington has a population of 8 million. With this model, assuming no one arrives from or leaves to anywhere else, what will the population settle down to over time?

\section{Section 2}

In problems 2--7, find the eigenvalues and eigenvectors of $A$ and the algebraic and geometric multiplicity of the eigenvalues. Then diagonalize $A$ if possible, using block diagonalization with at most $2 \times 2$ blocks if necessary.

2. $A = \left[\begin{array}{cc} -8& 10 \\ -5& 7 \end{array}\right]$.

3. $A = \left[\begin{array}{cc} -1& -1 \\ 10& 5 \end{array}\right]$.

4. $A = \left[\begin{array}{ccc} -4& -6& 12 \\ 9& 11& -18 \\ 3& 3& -4 \end{array}\right]$.

5. $A = \left[\begin{array}{ccc} -10& 4& -4 \\ -13& 4& -10 \\ 1& 0& 2 \end{array}\right]$

6. $A = \left[\begin{array}{cccc} 11& -6& -4& -8 \\ -4& 1& -12& -4 \\ -12& 6& 3& 8 \\ 20& -10& 0& -11 \end{array}\right]$.

7. $A = \left[\begin{array}{cccc} 10& 6& -7& 12 \\ -4& -3& 4& -12 \\ 7& 6& -4& 12 \\ -1& 0& 1& 3 \end{array}\right]$.

~\\

8. Let $A = BDB^{-1}$ be a diagonalized $n \times n$ matrix, so that the columns of $B$ are eigenvectors of $A$. Use this factorization to describe the eigenvectors of $A^T$ (remember that transposing a product reverses the order of the factors).

9. Give an example of an invertible $3 \times 3$ matrix that is not diagonalizable, and an example of a diagonalizable $3 \times 3$ matrix that is not invertible.


\end{document}