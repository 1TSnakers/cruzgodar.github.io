\documentclass{article}
\usepackage[utf8]{inputenc}
\usepackage[T1]{fontenc}
\usepackage{amsmath}
\usepackage{amsfonts}
\usepackage{amssymb}
\usepackage[dvipsnames]{xcolor}
\usepackage{enumitem}
\usepackage{titlesec}
\usepackage{graphicx}
\usepackage[total={6.5in, 9in}, heightrounded]{geometry}
\usepackage{hyperref}

\graphicspath{{graphics/}}
\setenumerate[0]{label=\alph*)}
\setlength{\parindent}{0pt}
\setlength{\parskip}{8pt}
\setlength\fboxsep{0pt}
\renewcommand{\baselinestretch}{1.6}
\titleformat{\section}
{\normalfont \Large \bfseries \centering}{}{0pt}{}

\newcommand{\s}[1]{{\color{violet} #1}}

\begin{document}

\Large Name: \rule{2in}{0.15mm} \hfill Homework 5 | Math 342 | Cruz Godar \vspace{4pt} \normalsize

\textit{Due Wednesday of Week 6 at the start of class}

Complete the following problems and submit them as a pdf to Canvas. 8 points are awarded for thoroughly attempting every problem, and I'll select three problems to grade on correctness for 4 points each. Enough work should be shown that there is no question about the mathematical process used to obtain your answers.

\section{Section 4}

In problems 1--3, use the Gram-Schmidt process to produce an orthonormal basis for the given subspace $X$. Then find the orthogonal decomposition of given vector $\vec{v}$ as $\vec{v} = \vec{x} + \vec{x}'$ for $\vec{x} \in X$ and $\vec{x'} \in X^\perp$.

1. $X = \operatorname{span}\left\{ \left[\begin{array}{c} 1 \\ 2 \\ 0 \end{array}\right], \left[\begin{array}{c} -1 \\ 1 \\ 4 \end{array}\right] \right\}$ and $\vec{v} = \left[\begin{array}{c} 1 \\ 2 \\ 3 \end{array}\right]$.

2. $X = \operatorname{span}\left\{ \left[\begin{array}{c} 2 \\ -4 \\ 1 \\ 3 \end{array}\right], \left[\begin{array}{c} 1 \\ 1 \\ 1 \\ 2 \end{array}\right], \left[\begin{array}{c} 0 \\ 1 \\ 2 \\ 6 \end{array}\right] \right\}$ and $\vec{v} = \left[\begin{array}{c} -1 \\ 1 \\ 0 \\ 3 \end{array}\right]$.

3. $X = \operatorname{span}\left\{ \left[\begin{array}{c} 0 \\ 1 \\ 0 \\ 1 \end{array}\right], \left[\begin{array}{c} 1 \\ 1 \\ 0 \\ 0 \end{array}\right]\right\}$ and $\vec{v} = \left[\begin{array}{c} -1 \\ 1 \\ 0 \\ 3 \end{array}\right]$.

~\\

In problems 4--5, find the closest vector $\vec{x} \in X$ to $\vec{v}$, and compute the distance between the two.

4. $X = \operatorname{span}\left\{ \left[\begin{array}{c} 1 \\ 2 \\ 2 \end{array}\right] \right\}$ and $\vec{v} = \left[\begin{array}{c} 3 \\ 1 \\ 4 \end{array}\right]$.

5. $X = \operatorname{span}\left\{ \left[\begin{array}{c} 4 \\ 2 \\ 1 \end{array}\right], \left[\begin{array}{c} -1 \\ -1 \\ 1 \end{array}\right] \right\}$ and $\vec{v} = \left[\begin{array}{c} 1 \\ 0 \\ 0 \end{array}\right]$.

~\\

6. Let $A = \left[\begin{array}{cc} 1& -1 \\ 2& 1 \\ 0& 4 \end{array}\right]$ be the matrix whose columns are the basis vectors from problem 1. By using the data from the Gram-Schmidt process, write $A = QR$ for a $3 \times 2$ unitary matrix $Q$ and a $2 \times 2$ upper triangular matrix $R$ whose eigenvalues are all positive. This is known as a \textbf{$\mathbf{QR}$ factorization} of $A$ --- write a brief sentence explaining why this is always possible for a matrix with linearly independent columns.

7. Find a $QR$ factorization of $A = \left[\begin{array}{ccc} 2& 1& 0 \\ -4& 1& 1 \\ 1& 1& 2 \\ 3& 2& 6 \end{array}\right]$.

8. Find a $QR$ factorization of $A = \left[\begin{array}{cc} 0& 1 \\ 1& 1 \\ 0& 0 \\ 1& 0 \end{array}\right]$.

~\\

9. Let $X$ be a subspace of $\mathbb{R}^n$. What is the kernel of the map $\operatorname{proj}_X$? (This should be a brief answer.)

10. True or false: for any subspace $X$ of $\mathbb{R}^n$ and any $\vec{v} \in \mathbb{R}^n$, $\left| \left| \operatorname{proj}_X\left( \vec{v} \right) \right| \right| \leq \left| \left| \vec{v} \right| \right|$. If true, briefly explain why, and if not, provide a short counterexample.


\end{document}