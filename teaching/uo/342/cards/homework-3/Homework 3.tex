\documentclass{article}
\usepackage[utf8]{inputenc}
\usepackage[T1]{fontenc}
\usepackage{amsmath}
\usepackage{amsfonts}
\usepackage{amssymb}
\usepackage[dvipsnames]{xcolor}
\usepackage{enumitem}
\usepackage{titlesec}
\usepackage{graphicx}
\usepackage[total={6.5in, 9in}, heightrounded]{geometry}
\usepackage{hyperref}

\graphicspath{{graphics/}}
\setenumerate[0]{label=\alph*)}
\setlength{\parindent}{0pt}
\setlength{\parskip}{8pt}
\setlength\fboxsep{0pt}
\renewcommand{\baselinestretch}{1.6}
\titleformat{\section}
{\normalfont \Large \bfseries \centering}{}{0pt}{}

\newcommand{\s}[1]{{\color{violet} #1}}

\begin{document}

\Large Name: \rule{2in}{0.15mm} \hfill Homework 3 | Math 341 | Cruz Godar \vspace{4pt} \normalsize

\textit{Due Wednesday of Week 4 at the start of class}

Complete the following problems and submit them as a pdf to Canvas. 8 points are awarded for thoroughly attempting every problem, and I'll select three problems to grade on correctness for 4 points each. Enough work should be shown that there is no question about the mathematical process used to obtain your answers.

\section{Section 3}

In problems 1--3, find the general solution to the system of differential equations.

1.

\begin{align*}
	x_1' &= 2x_1 - x_2\\
	x_2' &= 3x_1 - 2x_2.
\end{align*}

For this problem, also sketch a vector field.

2.

\begin{align*}
	x_1' &= 19x_1 - 4x_2 + 8x_3\\
	x_2' &= -8x_1 + 5x_2 - 10x_3\\
	x_3' &= -x_1 - 2x_2 + 4x_3.
\end{align*}

3.

\begin{align*}
	x' &= 2x + 2y - 2z\\
	y' &= -3x + 7y + 3z\\
	z' &= -5x + 5y + 5z.
\end{align*}

~\\

4. Two tanks are set up in a cyclical cascade. Tank 1 initially contains $100$ gallons of water and $10$ pounds of sugar, and tank 2 initially contains $50$ gallons of water and no sugar. At time $t = 0$, two valves are opened --- the well-mixed solution in tank 1 flows into tank 2 at a rate of $5$ gallons per second, and the well-mixed solution in tank 2 flows \textit{back} into tank 1 at $5$ gallons per second. After one minute, what is the concentration of sugar in each tank?

~\\

5. Let $f(\lambda) = \lambda^2 + a\lambda + b$ be a polynomial. The \textbf{companion matrix} to $f$ is the $2 \times 2$ matrix

\begin{align*}
	C(f) = \left[\begin{array}{cc}0& -b \\ 1& -a\end{array}\right].
\end{align*}

Show that the characteristic polynomial $\chi_{C(f)}$ is $f$.

6. Using companion matrices, create $2 \times 2$ matrices with real entries and the following eigenvalues:

\begin{enumerate}

	\item $\lambda_1 = 1, \quad \lambda_2 = 1$.

	\item $\lambda_1 = 1, \quad \lambda_2 = -1$.

	\item $\lambda_1 = -1, \quad \lambda_2 = -1$.

	\item $\lambda_1 = 1 + 2i, \quad \lambda_2 = 1 - 2i$.

	\item $\lambda_1 = -1 + 2i, \quad \lambda_2 = -1 - 2i$.

	\item $\lambda_1 = i, \quad \lambda_2 = -i$.

\end{enumerate}

7. For each of the six matrices in the previous problem, sketch a vector field for $-2 \leq x \leq 2$ and $-2 \leq y \leq 2$. Check your answer with the \href{https://cruzgodar.com/applets/vector-fields}{vector fields applet} --- for example, to plot

\begin{align*}
	x' &= 2x - y\\
	y' &= x + 4y,
\end{align*}

enter <code>(2x - y, x + 4y)</code> in the box and hit generate.


\end{document}