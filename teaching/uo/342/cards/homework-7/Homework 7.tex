\documentclass{article}
\usepackage[utf8]{inputenc}
\usepackage[T1]{fontenc}
\usepackage{amsmath}
\usepackage{amsfonts}
\usepackage{amssymb}
\usepackage[dvipsnames]{xcolor}
\usepackage{enumitem}
\usepackage{titlesec}
\usepackage{graphicx}
\usepackage[total={6.5in, 9in}, heightrounded]{geometry}
\usepackage{hyperref}

\graphicspath{{graphics/}}
\setenumerate[0]{label=\alph*)}
\setlength{\parindent}{0pt}
\setlength{\parskip}{8pt}
\setlength\fboxsep{0pt}
\renewcommand{\baselinestretch}{1.6}
\titleformat{\section}
{\normalfont \Large \bfseries \centering}{}{0pt}{}

\newcommand{\s}[1]{{\color{violet} #1}}

\begin{document}

\Large Name: \rule{2in}{0.15mm} \hfill Homework 7 | Math 341 | Cruz Godar \vspace{4pt} \normalsize

\textit{Due Wednesday of Week 8 at the start of class}

Complete the following problems and submit them as a pdf to Canvas. 8 points are awarded for thoroughly attempting every problem, and I'll select three problems to grade on correctness for 4 points each. Enough work should be shown that there is no question about the mathematical process used to obtain your answers.

\section{Section 7}

1. Let $A$ be a real $m \times n$ matrix. Explain why $A^T\!A$ is guaranteed to be diagonalizable.

2. Let $A$ be a symmetric (i.e. $A^T = A$), $n \times n$, real matrix with all positive eigenvalues and define $\left< \vec{v}, \vec{w} \right> = \vec{v}^T A \vec{w}$ for $\vec{v}, \vec{w} \in \mathbb{R}^n$.

\begin{enumerate}

	\item Show from the formula that this is symmetric --- that $\left< \vec{v}, \vec{w} \right> = \left< \vec{w}, \vec{v} \right>$.

	\item Show that it's also bilinear --- that $\left< c\vec{u} + \vec{v}, \vec{w} \right> = c\left< \vec{u}, \vec{w} \right> + \left< \vec{v}, \vec{w} \right>$.

	\item Setting $\vec{v} = \vec{w} = \vec{0}$, we have that $\left< \vec{0}, \vec{0} \right> = 0$. Explain why for any nonzero vector $\vec{v} \in \mathbb{R}^n$, $\left< \vec{v}, \vec{v} \right> > 0$.

\end{enumerate}

\section{Section 8}

In problems 3--5, write the matrix as $A = BJB^{-1}$ for a matrix $J$ in Jordan normal form.

3. $A = \left[\begin{array}{ccc} 0& -5& -2 \\ 2& 6& 1 \\ -2& -3& 2 \end{array}\right]$.

4. $A = \left[\begin{array}{ccc} -3& -1& -2 \\ -1& -1& -1 \\ 2& 1& 1 \end{array}\right]$.

5. $A = \left[\begin{array}{ccc} 12& 4& 4 \\ -24& -8& -8 \\ -3& -1& 0 \end{array}\right]$.


\end{document}