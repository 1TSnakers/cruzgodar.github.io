\documentclass{article}
\usepackage[utf8]{inputenc}
\usepackage[T1]{fontenc}
\usepackage{amsmath}
\usepackage{amsfonts}
\usepackage{amssymb}
\usepackage[dvipsnames]{xcolor}
\usepackage{enumitem}
\usepackage{titlesec}
\usepackage{graphicx}
\usepackage[total={6.5in, 9in}, heightrounded]{geometry}
\usepackage{hyperref}

\graphicspath{{graphics/}}
\setenumerate[0]{label=\alph*)}
\setlength{\parindent}{0pt}
\setlength{\parskip}{8pt}
\setlength\fboxsep{0pt}
\renewcommand{\baselinestretch}{1.6}
\titleformat{\section}
{\normalfont \Large \bfseries \centering}{}{0pt}{}

\newcommand{\s}[1]{{\color{violet} #1}}

\begin{document}

\Large Name: \rule{2in}{0.15mm} \hfill Homework 5 | Math 253 | Cruz Godar \vspace{4pt} \normalsize

\textit{Due Wednesday of Week 6 at the start of class}

Complete the following problems and submit them as a pdf to Canvas. 8 points are awarded for thoroughly attempting every problem, and I'll select three problems to grade on correctness for 4 points each. Enough work should be shown that there is no question about the mathematical process used to obtain your answers.

~\\

In problems 1--12, determine if the series converges absolutely, converges conditionally, or diverges. For alternating series that converge, estimate the series to within $0.1$ of its actual value.

1. $\displaystyle \sum_{n = 1}^\infty \frac{(-1)^n\ln(n)}{n}$.

2. $\displaystyle \sum_{k = 2}^\infty \frac{(-1)^{k + 1}}{\ln(\ln(k))}$.

3. $\displaystyle \sum_{n = 1}^\infty \frac{2^n}{n!}$.

4. $\displaystyle \sum_{n = 1}^\infty \left( \frac{2n + 1}{3n + 2} \right)^n$.

5. $\displaystyle \sum_{k = 1}^\infty \frac{(-1)^{k + 1}}{k!}$.

6. $\displaystyle \sum_{n = 2}^\infty \frac{2^n}{\ln(n)}$.

7. $\displaystyle \sum_{n = 1}^\infty \frac{(n!)^2}{(2n)!}$.

8. $\displaystyle \sum_{i = 1}^\infty \frac{\sin(i)}{i\sqrt{i}}$.

9. $\displaystyle \sum_{n = 1}^\infty \frac{n^2}{2^n}$.

10. $\displaystyle \sum_{m = 1}^\infty \frac{(-1)^m \sin\left( \frac{1}{m} \right)}{m}$.

11. $\displaystyle \sum_{n = 1}^\infty (-1)^n \left( 1 - n^{1/n} \right)$.

12. $\displaystyle \sum_{k = 1}^\infty \left( k^{1/k} - 1 \right)^k$.

~\\

The ratio and root tests look pretty similar to one another --- let's see if we can find a relationship between the two.

13. Let $\displaystyle \sum_{n = 1}^\infty a_n$ be a series with positive terms that converges via the ratio test: i.e. $\displaystyle \lim_{n \to \infty} \frac{a_{n + 1}}{a_n} < 1$. Therefore, there is a value $r < 1$ so that whenever $n \geq N$ for some $N$, $\frac{a_{n + 1}}{a_n} \leq r$.

\begin{enumerate}

	\item Show that when $n \geq N$, $a_n \leq a_N r^{n - N}$.

	\item Raise both sides of the inequality to the power of $\frac{1}{n}$ to get $a_n^{1/n} \leq a_N^{1/n} r^{1 - \frac{N}{n}}$. Now take the limit of both sides as $n \to \infty$.

	\item What can we conclude about the relationship between ratio test and the root test?

\end{enumerate}

14. Let $\displaystyle a_n = \frac{1}{2^{n + (-1)^{n + 1}}}$ and consider the sum

$$
	\sum_{n = 1}^\infty a_n = \frac{1}{2^1} + \frac{1}{2^0} + \frac{1}{2^3} + \frac{1}{2^2} + \frac{1}{2^5} + \frac{1}{2^4} + \frac{1}{2^7} + \frac{1}{2^6} + \cdots.
$$

\begin{enumerate}

	\item What does the ratio test say about this series? (Hint: treat the cases when $n$ is even and odd separately.)

	\item What does the root test say?

	\item What can we conclude about the relationship between ratio test and the root test?

\end{enumerate}


\end{document}