\documentclass{article}
\usepackage[utf8]{inputenc}
\usepackage[T1]{fontenc}
\usepackage{amsmath}
\usepackage{amsfonts}
\usepackage{amssymb}
\usepackage[dvipsnames]{xcolor}
\usepackage{enumitem}
\usepackage{titlesec}
\usepackage{graphicx}
\usepackage[total={6.5in, 9in}, heightrounded]{geometry}
\usepackage{hyperref}

\graphicspath{{graphics/}}
\setenumerate[0]{label=\alph*)}
\setlength{\parindent}{0pt}
\setlength{\parskip}{8pt}
\setlength\fboxsep{0pt}
\renewcommand{\baselinestretch}{1.6}
\titleformat{\section}
{\normalfont \Large \bfseries \centering}{}{0pt}{}

\newcommand{\s}[1]{{\color{violet} #1}}

\begin{document}

\Large Name: [YOUR NAME HERE] \hfill Homework 1 | Math 253 | Cruz Godar \vspace{4pt} \normalsize

\textit{Due Wednesday of Week 2 at the start of class}

Complete the following problems and submit them as a pdf to Canvas. 8 points are awarded for thoroughly attempting every problem, and I'll select three problems to grade on correctness for 4 points each. Enough work should be shown that there is no question about the mathematical process used to obtain your answers. 

~\\

In problems 1--6, write the first five terms of the sequence and find an explicit formula for the $n$th term of the sequence if it is not already given.

1. $\displaystyle a_n = \left( \frac{1}{2} \right)^n$.

2. $\displaystyle b_n = \cos\left( \frac{\pi}{2} n \right)$.

3. $\displaystyle c_1 = 1$ and $\displaystyle c_n = nc_{n - 1}$ for $n \geq 2$.

4. $\displaystyle d_1 = \frac{1}{5}$, $\displaystyle d_2 = \frac{1}{5}$, and $d_n = d_{n - 1}d_{n - 2}$ for $n \geq 3$. (You may state your explicit formula in terms of another sequence).

5. $e_1 = 6$, $e_2 = 2$, and $(e_n)$ is an arithmetic sequence.

6. $f_1 = 6$, $f_2 = 2$, and $(f_n)$ is a geometric sequence.

7. For each of the sequences in problems 1--6, find the limit if it exists. If it does exist, find a positive integer $N$ so that all terms of the sequence with index at least $N$ are within $\varepsilon = 0.1$ of the limit.

~\\

In problems 8--12, determine if the sequence converges and find the limit of the sequence if it does. Justify your answers.

8. $\displaystyle a_n = \frac{3n^3 - 2n^2}{n^3 + 1}$.

9. $\displaystyle b_n = \frac{(-1)^n}{\sqrt{n}}$.

10. $\displaystyle c_n = \tan(n)$.

11. $\displaystyle d_n = \frac{2^n}{n!}$.

12. $\displaystyle e_n = \frac{n^n}{n!}$.

~\\

13. Give an example of a sequence that is monotone increasing that does not converge, and a sequence that is bounded below but does not converge.

14. If a sequence is not bounded above, can it converge? Explain.

15. If a sequence $a_n$ has infinitely many positive terms \textit{and} infinitely many negative terms, can it still converge? If so, are there restrictions on what it can converge to?

16. Suppose $(a_n)$ is a sequence of rational numbers with $(a_n) \to a$. Is $a$ necessarily a rational number?

17. Let $a_n$ be a sequence and let $b_n = \left| a_{n + 1} - a_n \right|$. If $(b_n) \to 0$, does $(a_n)$ have to converge?


\end{document}