\documentclass{article}
\usepackage[utf8]{inputenc}
\usepackage[T1]{fontenc}
\usepackage{amsmath}
\usepackage{amsfonts}
\usepackage{amssymb}
\usepackage[dvipsnames]{xcolor}
\usepackage{enumitem}
\usepackage{titlesec}
\usepackage{graphicx}
\usepackage[total={6.5in, 9in}, heightrounded]{geometry}
\usepackage{hyperref}

\graphicspath{{graphics/}}
\setenumerate[0]{label=\alph*)}
\setlength{\parindent}{0pt}
\setlength{\parskip}{8pt}
\setlength\fboxsep{0pt}
\renewcommand{\baselinestretch}{1.6}
\titleformat{\section}
{\normalfont \Large \bfseries \centering}{}{0pt}{}

\newcommand{\s}[1]{{\color{violet} #1}}

\begin{document}

\Large Name: \rule{2in}{0.15mm} \hfill Homework 8 | Math 253 | Cruz Godar \vspace{4pt} \normalsize

\textit{Due Wednesday of Week 10 at the start of class}

Complete the following problems and submit them as a pdf to Canvas. 8 points are awarded for thoroughly attempting every problem, and I'll select three problems to grade on correctness for 4 points each. Enough work should be shown that there is no question about the mathematical process used to obtain your answers.

~\\

In problems 1--5, find a Taylor series for $f(x)$ centered at $x = a$ and determine its interval of convergence.

1. $\displaystyle f(x) = \cos(x)$, $\displaystyle a = 0$.

2. $\displaystyle f(x) = \sin(2x)$, $\displaystyle a = \frac{\pi}{2}$.

3. $\displaystyle f(x) = \ln(-x)$, $\displaystyle a = -1$.

4. $\displaystyle f(x) = \frac{1}{x}$, $\displaystyle a = 1$.

5. $\displaystyle f(x) = \sqrt[3]{x + 1}$, $\displaystyle a = 0$.

~\\

In problems 6--9, compute the value of the series and justify your answer.

6. $\displaystyle \sum_{n = 0}^\infty \frac{2^n}{n!}$.

7. $\displaystyle \sum_{n = 0}^\infty \frac{(-1)^n}{n + 1}$.

8. $\displaystyle \sum_{n = 0}^\infty (n + 1)\left( \frac{1}{2} \right)^n$.

9. $\displaystyle \sum_{n = 0}^\infty \frac{(-4)^n}{(2n)!}$.

~\\

In problems 10--13, approximate the value to within $0.01$ using Taylor series. 

10. $\displaystyle \cos(2)$.

11. $\displaystyle \frac{1}{e^2}$.

12. $\displaystyle \ln\left( \frac{1}{2} \right)$.

13. $\displaystyle \sqrt[4]{\frac{1}{2}}$.

~\\

14. Using the fact that $\displaystyle \sum_{n = 1}^\infty \frac{1}{n^2} = \frac{\pi^2}{6}$, evaluate $\displaystyle \int_0^1 \frac{\ln(1 + x)}{x}$.

15. Find a series solution to $\displaystyle \frac{1}{x} + y'' = \frac{y}{x}$.

~\\

16. The function $\sin(x)$ has zeroes at exactly $x = \pi n$ for integers $n$, and this allows us to write $\sin(x)$ as

$$
	\sin(x) = x\left( 1 - \frac{x}{\pi} \right)\left( 1 + \frac{x}{\pi} \right)\left( 1 - \frac{x}{2\pi} \right)\left( 1 + \frac{x}{2\pi} \right) \cdots.
$$

Note that this isn't true for functions in general! There can often be a lot more going on with a function than we can determine by its zeros.

\begin{enumerate}

	\item Divide by $x$ to find an expression for $\displaystyle \frac{\sin(x)}{x}$.

	\item Use the difference of squares formula $(a - b)(a + b) = a^2 - b^2$ to group each two consecutive factors in the product.

	\item Expand $\displaystyle \frac{\sin(x)}{x}$ as a Maclaurin series. What is the coefficient of $x^2$?

	\item In the grouped product, what is the coefficient of $x^2$? You'll likely need to express it as a sum.

	\item Set the two sides equal to one another. What have you shown?

\end{enumerate}


\end{document}