\documentclass{article}
\usepackage[utf8]{inputenc}
\usepackage[T1]{fontenc}
\usepackage{amsmath}
\usepackage{amsfonts}
\usepackage{amssymb}
\usepackage[dvipsnames]{xcolor}
\usepackage{enumitem}
\usepackage{titlesec}
\usepackage{graphicx}
\usepackage[total={6.5in, 9in}, heightrounded]{geometry}
\usepackage{hyperref}

\graphicspath{{graphics/}}
\setenumerate[0]{label=\alph*)}
\setlength{\parindent}{0pt}
\setlength{\parskip}{8pt}
\setlength\fboxsep{0pt}
\renewcommand{\baselinestretch}{1.6}
\titleformat{\section}
{\normalfont \Large \bfseries \centering}{}{0pt}{}

\newcommand{\s}[1]{{\color{violet} #1}}

\begin{document}

\Large Name: \rule{2in}{0.15mm} \hfill Homework 7 | Math 253 | Cruz Godar \vspace{4pt} \normalsize

\textit{Due Wednesday of Week 8 at the start of class}

Complete the following problems and submit them as a pdf to Canvas. 8 points are awarded for thoroughly attempting every problem, and I'll select three problems to grade on correctness for 4 points each. Enough work should be shown that there is no question about the mathematical process used to obtain your answers.

~\\

In problems 1--6, find the interval and radius of convergence of the power series.

1. $\displaystyle \sum_{n = 1}^\infty \frac{x^n}{n}$.

2. $\displaystyle \sum_{n = 1}^\infty \frac{x^n}{n^2}$.

3. $\displaystyle \sum_{n = 1}^\infty nx^n$.

4. $\displaystyle \sum_{n = 1}^\infty \frac{x^n}{2^n}$.

5. $\displaystyle \sum_{n = 1}^\infty \frac{2^n x^n}{n!}$.

6. $\displaystyle \sum_{n = 2}^\infty \frac{x^n}{\ln(n)}$.

~\\

7. Find the radius of convergence of $\displaystyle \sum_{n = 0}^\infty \frac{(n!)^3 x^n}{(3n)!}$, but not the interval (i.e. you don't need to check the endpoints).

~\\

In problems 8--15, expand each function $f(x)$ as a power series and find its interval and radius of convergence.

8. $\displaystyle \frac{1}{1 - x}$.

9. $\displaystyle \frac{1}{1 - x^3}$.

10. $\displaystyle \frac{1}{1 - 2x}$.

11. $\displaystyle \frac{1}{x}$.

12. $\displaystyle \frac{x^4}{1 - x^2}$.

13. $\displaystyle \frac{1}{2 - x}$.

14. $\displaystyle \frac{1}{x}$.

15. $\displaystyle \frac{x}{1 - (1 - x)}$. Does your result make sense?

~\\

16. Give examples of power series with intervals of convergence of

\begin{enumerate}

	\item $(-1, 1]$.

	\item $[-4, 0]$.

	\item Only $x = 5$ and no other numbers.

\end{enumerate}

17. If a power series converges at $x = 1$, does it also have to converge at $x = 0$? Why or why not?

18. Is it possible to express a function as a power series in more than one way? For example, can we express $\displaystyle \frac{1}{1 - x}$ as a power series centered at $x = \frac{1}{2}$? What happens to the interval of convergence if so? What about as a series centered at $x = 1$?


\end{document}