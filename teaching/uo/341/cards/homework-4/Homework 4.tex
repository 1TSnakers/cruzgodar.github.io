\documentclass{article}
\usepackage[utf8]{inputenc}
\usepackage[T1]{fontenc}
\usepackage{amsmath}
\usepackage{amsfonts}
\usepackage{amssymb}
\usepackage[dvipsnames]{xcolor}
\usepackage{enumitem}
\usepackage{titlesec}
\usepackage{graphicx}
\usepackage[total={6.5in, 9in}, heightrounded]{geometry}
\usepackage{hyperref}

\graphicspath{{graphics/}}
\setenumerate[0]{label=\alph*)}
\setlength{\parindent}{0pt}
\setlength{\parskip}{8pt}
\setlength\fboxsep{0pt}
\renewcommand{\baselinestretch}{1.6}
\titleformat{\section}
{\normalfont \Large \bfseries \centering}{}{0pt}{}

\newcommand{\s}[1]{{\color{violet} #1}}

\begin{document}

\Large Name: \rule{2in}{0.15mm} \hfill Homework 4 | Math 341 | Cruz Godar \vspace{4pt} \normalsize

\textit{Due Wednesday of Week 5 at the start of class}

Complete the following problems and submit them as a pdf to Canvas. 8 points are awarded for thoroughly attempting every problem, and I'll select three problems to grade on correctness for 4 points each. Enough work should be shown that there is no question about the mathematical process used to obtain your answers.

\section{Section 5}

In problems 1--6, determine if the linear transformation $T$ is one-to-one, if it is onto, and if it is invertible. If it is invertible, find the inverse transformation.

1. $T: \mathbb{R}^2 \to \mathbb{R}^2$ defined by $T\left(\left[\begin{array}{c} x \\ y \end{array}\right]\right) = \left[\begin{array}{c} 2x + y \\ x + 2y\end{array}\right]$.

2. $T: \mathbb{R}^3 \to \mathbb{R}^2$ defined by $T\left(\left[\begin{array}{c} x \\ y \\ z \end{array}\right]\right) = \left[\begin{array}{c} x \\ y + z \end{array}\right]$.

3. $T: \mathbb{R}^2 \to \mathbb{R}^2$, where

$$
	T\left(\left[\begin{array}{c} 2 \\ 0 \end{array}\right]\right) = \left[\begin{array}{c} 2 \\ 4 \end{array}\right] \qquad T\left(\left[\begin{array}{c} 4 \\ 6 \end{array}\right]\right) = \left[\begin{array}{c} 4 \\ -10 \end{array}\right].
$$

4. $T: \mathbb{R}^3 \to \mathbb{R}^3$, where

$$
	T\left(\left[\begin{array}{c} 1 \\ 1 \\ 1 \end{array}\right]\right) = \left[\begin{array}{c} 3 \\ 7 \\ 4 \end{array}\right] \qquad T\left(\left[\begin{array}{c} 1 \\ 2 \\ 0 \end{array}\right]\right) = \left[\begin{array}{c} 1 \\ -1 \\ 3 \end{array}\right] \qquad T\left(\left[\begin{array}{c} 0 \\ 1 \\ 3 \end{array}\right]\right) = \left[\begin{array}{c} 6 \\ 16 \\ 7 \end{array}\right].
$$

5. $T: \mathbb{R}^3 \to \mathbb{R}$, where

$$
	T\left(\left[\begin{array}{c} 1 \\ -1 \\ 1 \end{array}\right]\right) = -3 \qquad T\left(\left[\begin{array}{c} 1 \\ 0 \\ 1 \end{array}\right]\right) = 1 \qquad T\left(\left[\begin{array}{c} 0 \\ 1 \\ 1 \end{array}\right]\right) = 4.
$$

6. $T: \mathbb{R}^5 \to \mathbb{R}^2$, where

$$
	T\left(\left[\begin{array}{c} 1 \\ 0 \\ 2 \\ 3 \\ 0 \end{array}\right]\right) = \left[\begin{array}{c} 1 \\ 1 \end{array}\right] \qquad T\left(\left[\begin{array}{c} 2 \\ 1 \\ 2 \\ -1 \\ 3 \end{array}\right]\right) = \left[\begin{array}{c} 2 \\ 0 \end{array}\right].
$$

~\\

7. Let $T : \mathbb{R}^4 \to \mathbb{R}^3$ be defined by the matrix $\left[\begin{array}{cccc} 1& 7& 2& -1 \\ -3& 6& 3& 12 \\ 0& 3& 1& 1 \end{array}\right]$.

\begin{enumerate}

	\item Is $T$ one-to-one? Is it onto? Is it invertible?

	\item Find the vectors $\vec{v}$ for which $T(\vec{v}) = \vec{0}$ and express it as a span of one or more vectors, i.e. $\operatorname{span}\{ \vec{v_1}, ..., \vec{v_n} \}$.

	\item Find vectors $\vec{w_1}, ..., \vec{w_m} \in \mathbb{R}^4$ that are linearly independent to $\vec{v_1}, ..., \vec{v_n}$ so that all together,

\end{enumerate}

$$
	\operatorname{span}\{ \vec{v_1}, ..., \vec{v_n}, \vec{w_1}, ..., \vec{w_m} \} = \mathbb{R}^4.
$$

<span style="width: 32px"></span>Hint: $n + m$ should equal $4$. (Why?)

\begin{enumerate}

	\item Show that $T(\vec{w_1}), ..., T(\vec{w_m})$ are linearly independent. Briefly explain why this has to be the case.

\end{enumerate}

~\\

8. Let $R : \mathbb{R}^2 \to \mathbb{R}^2$ be a function (not necessarily a linear transformation) defined by rotating its inputs $90^\circ$ counterclockwise. For example, $R\left( \left[\begin{array}{c} 1 \\ 0 \end{array}\right] \right) = \left[\begin{array}{c} 0 \\ 1 \end{array}\right]$ and $R\left( \left[\begin{array}{c} \frac{1}{2} \\ \frac{\sqrt{3}}{2} \end{array}\right] \right) = \left[\begin{array}{c} -\frac{\sqrt{3}}{2} \\ \frac{1}{2} \end{array}\right]$.

\begin{enumerate}

	\item Explain with a picture why $R$ is in fact a linear transformation, i.e. $R(\vec{v_1} + \vec{v_2}) = R(\vec{v_1}) + R(\vec{v_2})$ and $R(c\vec{v}) = cR(\vec{v})$.

	\item Find the matrix for $R$.

	\item Let $R_\theta : \mathbb{R}^2 \to \mathbb{R}^2$ be the linear transformation that rotates its inputs an angle $\theta$ counterclockwise, where $\theta$ is a variable. Find a matrix for $R_\theta$.

\end{enumerate}


\end{document}